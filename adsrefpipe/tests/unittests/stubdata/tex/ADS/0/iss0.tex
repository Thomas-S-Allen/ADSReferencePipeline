%R 0000nla.....0.....Z
%J Proceedings of the NASA LAW 2006, held February 14-16, 2006, UNLV, Las Vegas.
Published by NASA Ames, Moffett Field, CA, 2006., p.150
%D 00/2006
%T Difficulties in Laboratory Studies and Astronomical Observations of Organic Molecules: Hydroxyacetone and Lactic Acid
%A Apponi, A. J.; Brewster, M. A.; Hoy, J.; Ziurys, L. M.
%F AA(Life and Planets Astrobiology Center, The University of Arizona, Tucson, AZ, 85721 <EMAIL>aapponi@as.arizona.edu</EMAIL>),
AB(Life and Planets Astrobiology Center, The University of Arizona, Tucson, AZ, 85721; Current address: SFRYI Inc., Seattle, WA, 98109),
AC(Life and Planets Astrobiology Center, The University of Arizona, Tucson, AZ, 85721),
AD(Life and Planets Astrobiology Center, The University of Arizona, Tucson, AZ, 85721)
%B For the past 35 years, radio astronomy has revealed a rich organic chemistry in the interstellar gas, which is exceptionally complex towards active star-forming regions. New solar systems condense out of this gas and may influence the evolution of life on newly formed planets. Much of the biologically important functionality is present among the some 130 gas-phase molecules found to date, including alcohols, aldehydes, ketones, acids, amines, amides and even the simplest sugar - glycolaldehyde. Still, many unidentified interstellar radio signals remain, and their identification relies on further laboratory study.

The molecules hydroxyacetone and lactic acid are relatively small organic molecules, but possess rather complex rotational spectra owing to their high asymmetry.  Hydroxyacetone is particularly problematic because it possess a very low barrier to internal rotation, and exhibits strong coupling of the free-rotor states with the overall rotation of the molecule.  As in the case of acetamide, a full decomposition method was employed to order the resultant eigenstates onto normal asymmetric top eigenvectors.
%Z  \\bibitem[]{ref1} Braakman et al.  2005, 60th Int. Symp. Mol. Spectr., Ohio State University
\\bibitem[]{ref2} Halfen et al.  2006, \apj, 639, 237
\\bibitem[]{ref3} Hollis et al.  2000, \apj, 540, L107
\\bibitem[]{ref4} Ilyushin  2004, J. Mol. Spectr., 227, 140
\\bibitem[]{ref5} Kattija-Ari and Harmony  1980, Int. J. Quan. Chem., 14, 443
\\bibitem[]{ref6} Pszczokowski et al.  2005, J. Mol. Spectr., 234, 106
%I ORIGINAL: YES

%R 0000waas....0.....Z
%Z
\\bibitem{Bender90} Bender, R., A\&A, {\bf 229}, 441
(1990).
\\bibitem{Bender94} Bender, R., Saglia, R.P., and Gerhard, O.E., MNRAS, {\bf 269},
785 (1994).
\\bibitem{Bonfanti99} Bonfanti, P., Simien, F., Rampazzo, R. and Prugniel, Ph., 1999, A\&AS, {\bf 139},
483 (1999).
\\bibitem{Combes95} Combes, F., Rampazzo, R., Bonfanti, P.P., Prugniel, P. and Sulentic, J.W., A\&A, {\bf 297},
37 (1995).
\\bibitem{Coziol98} Coziol, R., Ribeiro, A. L. B., De Carvalho, R. and Capelato, H. V., ApJ, {\bf 493},
563 (1998).
\\bibitem{Fukazawa01} Fukazawa, Y., Nakazawa, K., Isobe, N., Ohashi, T. and Kamae, T., ApJ, {\bf 546},
87 (2001).
\\bibitem{Fukugita95} Fukugita, M., Shimasaku, K., Ichikawa T., Publications of the astronomical society of the Pacific, {\bf 107},
945 (1995).
\\bibitem{Mendes03} Mendes de Oliveira C., Aram, P., Plana, H. and Balkowski, C., AJ, {\bf 126},
2635 (2003).
\\bibitem{Nishiura00} Nishiura, S., Shimada, M., Ohyama, Y., Murayama, T. and Taniguchi, Y., AJ, {\bf 120},
1691 (2000).
\\bibitem{Pildis96} Pildis, R. A., Evrard, A. E., and Bergman, J. N., AJ, {\bf 112},
378 (1996).
\\bibitem{Rubin91} Rubin V.C., Hunter D.A., Ford W.K.Jr., ApJS, {\bf 76}, 153 (1991).
\\bibitem{Vennik93} Vennik, J., Richter, G. M. and Longo, G., AN, {\bf 314},
393 (1993).
\\bibitem{Vrtilek02} Vrtilek, J., M., Grego, L., David, L. P. et al., APS meeting,
B17.107 (2002).
\\bibitem{Zepf91} Zepf, S. E., Whitmore, B. C., Levison, H. F., ApJ, {\bf 383},
524 (1991).


%R 0000tivo....0.....Z
%J Toward an International Virtual Observatory,
Proceedings of the ESO/ESA/NASA/NSF Conference
held in Garching, Germany, 10-14 June 2002.  Edited by
P.J. Quinn, and K.M. Gorski.  ESO Astrophysics Symposia.
Berlin: Springer, 2003, p. 337.
%D 00/2003
%T The Hamburg/RASS Catalogue of Optical Identifications
of ROSAT Bright Source X-Ray Sources
%A Zickgraf, F.-J.; Engels, D.; Hagen, H.-J.; Reimers, D.;
Voges, W.
%F AA(Hamburger Sternwarte, Gojenbergsweg 112, 21029 Hamburg, Germany),
AB(Hamburger Sternwarte, Gojenbergsweg 112, 21029 Hamburg, Germany),
AC(Hamburger Sternwarte, Gojenbergsweg 112, 21029 Hamburg, Germany),
AD(Hamburger Sternwarte, Gojenbergsweg 112, 21029 Hamburg, Germany),
AE(Max-Planck-Institut f&uuml;r extraterrestrische Forschung,
Postfach 1312, 85741 Garching)
%B We used digitized Schmidt direct and prism  plates taken for the  northern
hemisphere Hamburg Quasar Survey (HQS) to obtain optical identifications for all
high galactic latitude  X-ray sources  in the ROSAT
Bright Source Catalogue (RASS-BSC) at |b| &geq; 30&deg; and
&delta; &geq; 0&deg;. In this part of the sky the RASS-BSC
contains 5341 X-ray sources.  We found 51% extragalactic
counterparts (AGN, galaxies, clusters of galaxies), and 31% stellar sources.
3% are empty fields and for further 15% no unique identification could be given. The digitized
database of Schmidt plates and tools to use this database for identification work will be
implemented in the German Virtual Observatory (GAVO).
%Z
\\bibitem{zick:Badeetal98}
Bade N., Engels D., Voges W., et al.,  1998, A\&AS 127, 145

\\bibitem{zick:Hagenetal95}
Hagen H.-J., Groote D., Engels D., Reimers D.,1995, A\&AS 111, 195

\\bibitem{zick:Vogesetal99}
Voges W., Aschenbach B., Boller Th., et al.  1999, A\&A, 349, 389


%R 0000egcs....0.....Z
%T The High-Redshift Formation of Metal-Poor Globular Cluster Systems
%A Santos, Michael R.
%F AA(Theoretical Astrophysics, 130-33 Caltech, Pasadena CA 91125, USA)
%J Extragalactic Globular Cluster Systems, Proceedings of the ESO Workshop held in Garching, Germany, 27-30 August 2002, p. 348.
%D 00/2003
%B We present a formation scenario for metal-poor
globular cluster systems.  We assume old metal-poor globular clusters
are formed only prior to reionization (at z &sim;7) in merging
proto-galactic building blocks; the building blocks assemble
hierarchically into larger structures such as galaxies.  We
additionally assume that globular cluster formation is suppressed
immediately after reionization: subsequent globular cluster formation
results in a gap in the metallicity distributions of globular cluster
systems.  We describe generic predictions of the model with regard to
the ``blue specific frequency problem'' and the spatial distribution
of metal-poor globular cluster systems.  We conclude with a discussion
of the range of applications for this model.

%Z \\bibitem{bar99} R. Barkana, A. Loeb: ApJ, 523, 54 (1999)
\\bibitem{ben02} A. J. Benson, C. G. Lacey, C. M. Baugh, S. Cole,
C. S. Frenk: MNRAS, 333, 156 (2002)
\\bibitem{for00} D. A. Forbes, K. L. Masters, D. Minniti, P. Barmby: A&amp;A, 358, 471 (2000)
\\bibitem{geb99} K. Gebhardt, M. Kissler-Patig: AJ, 118, 1526 (1999)
\\bibitem{gne00} N. Y. Gnedin: ApJ, 535, 530 (2000)
\\bibitem{sant:har96} W. E. Harris: AJ, 112, 1487 (1996)
\\bibitem{ros88} E. I. Rosenblatt, S. M. Faber, G. R. Blumenthal: ApJ
330, 191 (1988)
\\bibitem{wes93} M. J. West: MNRAS 265, 755 (1993)
\\bibitem{zin85} R. Zinn: ApJ 293, 424 (1985)


%R 0000acfp....0.....Z
%T Dark Matter in Dwarf Elliptical Galaxies
%A Zeilinger, Werner W.; Dejonghe, Herwig; De Rijcke, Sven; Hau, George K. T.
%F AA(Institut f&uuml;r Astronomie, Universit&auml;t Wien, T&uuml;rkenschanzstra&szlig;e 17, A-1180 Wien, Austria), AB(Sterrenkundig Observatorium, Universiteit Gent, Krijgslaan 281, S9, B-9000 Gent, Belgium), AC(Sterrenkundig Observatorium, Universiteit Gent, Krijgslaan 281, S9, B-9000 Gent, Belgium), AD(ESO, Casilla 19001, Santiago de Chile 19, Chile)
%J Astronomy, Cosmology and Fundamental Physics, Proceedings of the ESO-CERN-ESA Symposium held in Garching, Germany, 4-7 March 2002, p. 495.
%D 00/2003
%B Results from an ESO Large Programme on the search for dark matter in
dwarf elliptical galaxies (dEs) are presented. Deep imaging and
long-slit spectroscopy was obtained for a sample of dEs ranging from
dE0 to dS0. Only few dEs are found to be flattened by rotation and
the majority is supported by pressure anisotropy. The typical M/L
ratios are found to be in the range of 3 to 9 solar units which are
consistent with predictions based on cold dark matter cosmological
scenarios for galaxy formation.

%Z \\bibitem{zeil:journ1} H. C. Ferguson, B. Binggeli: A&amp;A Rev. 6, 67 (1994)

\\bibitem{zeil:journ2} A. Dekel, J. Silk: ApJ 303, 39 (1986)

\\bibitem{zeil:journ3} B. Moore, G. Lake, N. Katz: ApJ 495, 139 (1998)

\\bibitem{zeil:journ4} S. De Rijcke, H. Dejonghe: MNRAS 298, 677 (1998)

\\bibitem{zeil:journ5} H. C. Ferguson, A. Sandage: AJ 100, 1 (1990)

\\bibitem{zeil:journ6} S. De Rijcke, H. Dejonghe, W. W. Zeilinger, G. K. T. Hau: ApJ 559, L21 (2001)



%R 0000adass...0.....Z
%Z
\reference
Hoffman, W.,
Hudson, J.,
Sharpe, R. K.,
Grossman, A. W.,
Morgan, J. A., \&
Teuben, P.~J.
1996, \adassv, 436.

\reference
 Plante, R.,
 Pound, M. W.,
 Mehringer, D.,
 Scott, S. L.,
 Beard, A.,
 Daniel, P.,
 Hobbs, R.,
 Kraybill, J. C.,
 Wright, M.,
 Leitch, E.,
 Amarnath, N. S.,
 Rauch, K. P., \&
 Teuben, P. J.
\  2003, \adassxii, 269

\reference
 Scott, S. L.,
 Hobbs, R.,
 Beard, A.,
 Daniel, P.,
 Mehringer, D.,
 Plante, R.,
 Kraybill, J. C.,
 Wright, M.,
 Leitch, E.,
 Amarnath, N. S.,
 Pound, M. W.,
 Rauch, K. P., \&
 Teuben, P. J.
  2003, \adassxii, 265

\reference
Welch, W. J. et~al.\ 1996, \pasp, 108, 93

\reference
Yu, T. 2001, \adassx, 495

%R 0000ASIB....0.....Z
%T Transfer of Cyclotron Radiation in Plasmas on Magnetic White Dwarfs
%A Zheleznyakov, V. V.; Serber, A. V.
%F AA(Department of Astrophysics and Cosmic Plasma Physics, Institute of Applied Physics of Russian Academy of Sciences, 46 Uljanov st., Nizhni Novgorod, 603950 Russia), AB(Department of Astrophysics and Cosmic Plasma Physics, Institute of Applied Physics of Russian Academy of Sciences, 46 Uljanov st., Nizhni Novgorod, 603950 Russia)
%J White Dwarfs, proceedings of the conference held at the Astronomical Observatory of Capodimonte, Napoli, Italy. http://www.na.astro.it/meetings/wd2002/wd.html. To be published by Kluwer. (NATO Science Series II -- Mathematics, Physics and Chemistry, Vol. 105, p. 213.)
%D 00/2003

%Z \bibitem{Wickramasinghe-2000}
Wickramasinghe, D. T. and Ferrario, L. 2000, PASP, 12, 873.
\bibitem{Zheleznyakov-1996a} Zheleznyakov, V. V. 1996, Radiation in Astrophysical Plasmas, Kluwer.
\bibitem{Serber-1990}
Serber, A. V. 1990, SvA, 4, 291.
\bibitem{Serber-1999} Serber, A. V.
1999, Radiophys. Quantum\ Electron., 2, 911.
\bibitem{Zheleznyakov-1999a} Zheleznyakov, V. V., Koryagin,
S. A. and Serber, A. V. 1999, AstL, 5, 437.
\bibitem{Zheleznyakov-1987} Zheleznyakov, V. V. and Litvinchuk, A. A. 1987, SvA, 1, 159.
\bibitem{Zheleznyakov-1991} Zheleznyakov, V. V. and Serber, A. V., 1991 SvAL, 7, 179.
\bibitem{Zheleznyakov-1994a} Zheleznyakov, V. V. and Serber, A. V. 1994, ApJS, 0, 783.
\bibitem{Zheleznyakov-1995} Zheleznyakov, V. V.
and Serber, A. V. 1995, AdSpR, 6, 77.
\bibitem{Zheleznyakov-1999b} Zheleznyakov, V. V., Koryagin, S. A.
and Serber, A. V. 1999, AstL, 5, 445.
\bibitem{Zheleznyakov-1986} Zheleznyakov, V. V. and Litvinchuk, A. A. 1986, Plasma Astrophysics, ESA SP-251,
p. 375.
\bibitem{Serber-2000} Serber, A. V. 2000, ARep, 4, 815.
\bibitem{Koryagin-2002} Zheleznyakov, V. V., Koryagin,
S. A. and Serber, A. V. 2002, this proceedings. \bibitem{Gospodchikov-2002} Gospodchikov, E. D. and Serber, A. V. 2002,
this proceedings.
\bibitem{Yampolsky-2001} Serber, A. V. and Yampol'sky, N. A.
2001, Radiophys. Quantum\ Electron., 4, 25.
\bibitem{Zheleznyakov-1996b} Zheleznyakov, V. V., Serber, A. V., and Kuijpers, J. 1996, A&amp;A, 08,
465.bibitem{Zheleznyakov-1994b} Zheleznyakov, V. V. and Serber, A. V. 1994, SSRv, 8,
275.bibitem{Serber-2002} Zheleznyakov, V. V. and Serber, A. V. 2002, this proceedings.


%R 0000mglh....0.....Z
%T The 2MASS Luminosity, Velocity, and Mass Functions of Galaxies
%A Pahre, M. A.; Kochanek, C. S.; Falco, E. E.; Huchra, J. P.
%F AA(Harvard--Smithsonian Center for Astrophysics, 60 Garden Street, Cambridge, MA 02138, USA), AB(Harvard--Smithsonian Center for Astrophysics, 60 Garden Street, Cambridge, MA 02138, USA), AC(Harvard--Smithsonian Center for Astrophysics, 60 Garden Street, Cambridge, MA 02138, USA), AD(Harvard--Smithsonian Center for Astrophysics, 60 Garden Street, Cambridge, MA 02138, USA)
%J The Mass of Galaxies at Low and High Redshift. Proceedings of the ESO Workshop held in Venice, Italy, 24-26 October 2001, p. 32.
%D 00/2003
%B The velocity function (VF) of galaxies is analogous to the luminosity function (LF), but
a more direct tool for comparing observations and theory.
The 2MASS LF is calculated here for a local sample of &sim;4000 galaxies,
and the VF is derived from a &sim;1000 galaxy subsample drawn from it.
The observed VF follows a Schechter functional form.
All current analytical, semi-analytical, and numerical galaxy formation models predict
power-law (or nearly so) shapes for the VF, and hence cannot reproduce the observations.
It remains an open challenge for any galaxy formation model to explain the observed VF.

%Z {\bibitem{2001AJ....121.2358B} M. R. Blanton, et al.:  \aj, 121, 2358 (2001) {blanton01}

{\bibitem{1989MNRAS.237.1127C} S. Cole, N. Kaiser: \mnras, 237, 1127 (1989) {cole89}

{\bibitem{2001MNRAS.326..255C} S. Cole, et al.: \mnras, 326, 255 (2001) {cole01}

{\bibitem{2001MNRAS.324..825C} N. Cross, et al.: \mnras, 324, 825 (2001) {cross01}

{\bibitem{fukugita91} M. Fukugita, Turner, E. L.: MNRAS, 253, 99 (1991) {fukugita91}

{\bibitem{gonzalez00} A. H. Gonzalez, et al.: ApJ, 528, 145 (2000) {gonzalez00}

{\bibitem{1993ApJ...419...12K}  C. S. Kochanek:\apj, 419, 12 (1993) {kochanek93}

{\bibitem{kochanek96} C. S. Kochanek:  ApJ, 466, 638 (1996) {kochanek96}

{\bibitem{2001ApJ...560..566K} C. S. Kochanek, et al.: \apj, 560, 566 (2001) {kochanek01}

{\bibitem{kochanek02} C. S. Kochanek, M. A. Pahre, E. E. Falco: ApJ, submitted (2002)
	(astro-ph/0011458) {kochanek02}

{\bibitem{kochanekwhite01} C. S. Kochanek, M. White:  ApJ, submitted (2002)
	(astro-ph/0102334) {kochanekwhite01}

{\bibitem{kochanek01stsci} C. S. Kochanek: in the proceedings of The Dark Universe
	(meeting at STScI, April 2-5, 2001) M. Livio, ed. (2002) (astro-ph/0108160) {kochanek01stsci}

{\bibitem{1996ApJ...464...60L} H. Lin, et al.: \apj, 464, 60 (1996) {lin96}

{\bibitem{Pahre et al. 2002} M. A. Pahre, C. S. Kochanek, E. E. Falco:  in preparation (2002) {pahre02}

{\bibitem{1976ApJ...203..297S} P. Schechter: \apj, 203, 297 (1976) {schechter76}

%R 0000fthp....0.....Z
%T The Supernova Program of the Canada-France-Hawaii-Telescope Legacy Survey
%A Pain, R.; the SNLS Collaboration
%F AA(LPNHE, Universit&eacute;s de Paris VI &amp; VII, 4, Place Jussieu, Tour 33 RdC, F-75252 Paris cedex, France, http://snls.in2p3.fr)
%J From Twilight to Highlight: The Physics of Supernovae.  Proceedings of the ESO/MPA/MPE Workshop held in Garching, Germany, 29-31 July 2002, p. 408.
%D 00/2003

%B The CFHT Legacy Survey (CFHTLS) is a large-scale multi-component
program intended to operate for 5 years beginning February 2003. The
CFHTLS will consist of 3 surveys: a very wide shallow survey, a wide
synoptic survey and a deep synoptic survey.  Covering 4 square degrees
in four MegaCam fields through the whole filter set (u*g'r'i'z') this
deep synoptic survey will be sequenced at an average rate of 5 nights
a dark run for 5 runs in a row each year on each field. It will make
it possible to observe each 1 square degree field every 2-3 nights in
4 colors for 5 months and aims primarily at detecting and monitoring
as many as 2000 SNe up to redshift z &sim;1.  This unprecedented
supernova data set will permit to obtain a precise measurement of the
cosmic star formation rate as well as of the cosmological parameters
including a first measurement of the equation of state of Dark Energy.

%Z \bibitem{Riess98}
Riess, A. et al. 1998, AJ. 116, 1009
\bibitem{Perlmutter99}
Perlmutter, S. et al. 1999, ApJ 517, 565
\bibitem{Garnavich98}
Garnavich, P. et al. 1998, ApJ. 509, 74
\bibitem{SNLS}
The ``SuperNova Legacy Survey''. See  http://snls.in2p3.fr/
\bibitem{Pain02}
Pain, R. et al. 2002, ApJ 577, 120
\bibitem{CFHTLS}
See  http://www.cfht.hawaii.edu/Science/CFHLS/
\bibitem{SNLSprop}
See  http://snls.in2p3.fr/notes/general/proposal_140901.ps


%R 0000evn.....0.....Z
%T Sub-arcsecond radio jets in the high mass X-ray binary LS I +61 303
%A Massi, M.; Rib&oacute;, M.; Paredes, J. M.; Peracaula, M.; Mart&iacute;, J.; Garrington, S. T.
%F AA(Max Planck Institut f&uuml;r Radioastronomie, Auf dem H&uuml;gel 69, 53121 Bonn, Germany), AB(Departament d'Astronomia i Meteorologia, Universitat de Barcelona, Av. Diagonal 647, 08028 Barcelona, Spain), AC(Departament d'Astronomia i Meteorologia, Universitat de Barcelona, Av. Diagonal 647, 08028 Barcelona, Spain), AD(Departament d'Astronomia i Meteorologia, Universitat de Barcelona, Av. Diagonal 647, 08028 Barcelona, Spain), AE(Departamento de F&iacute;sica, Escuela Polit&eacute;cnica Superior, Universidad de Ja&eacute;n, Calle Virgen de la Cabeza 2, 23071 Ja&eacute;n, Spain), AF(Nuffield Radio Astronomy Laboratories, Jodrell Bank, Macclesfield, Cheshire SK11 9DL, UK)
%J 6th European VLBI Network Symposium on New Developments in VLBI Science and Technology, held in Bonn, June 25th-28th 2002, proceedings edited by E. Ros, R. W. Porcas, A. P. Lobanov, and J. A. Zensus, published by the Max-Planck-Institut fuer Radioastronomie (Bonn). p. 279.
%D 06/2002
%B We present four runs of MERLIN observations of LS I +61 303 carried out from
February to May 2001. We have detected a jet extending up to &sim;400 AU,
aligned with the previously discovered structure of few AU found at higher
resolution with the EVN.

%Z
\bibitem[2002]{gregory02}
Gregory, P. C.
2002, ApJ, in press

\bibitem[1981]{hutchings81}
Hutchings, J. B., &amp; Crampton, D.
1981, PASP, 93, 486

\bibitem[2001]{massi01}
Massi, M., Rib&oacute;, M., Paredes, J. M., Peracaula, M., &amp; Estalella, R.
2001, A&amp;A, 376, 217

\bibitem[1999]{mirabel99}
Mirabel, I. F., &amp; Rodr&iacute;guez, L. F.
1999, ARA&amp;A, 37, 409

\bibitem[1991]{paredes91}
Paredes, J. M., Mart&iacute;, J., Estalella, R., &amp; Sarrate, J.
1991, A&amp;A, 248, 124

\bibitem[1982]{taylor82}
Taylor, A. R., &amp; Gregory, P. C.
1982, ApJ, 255, 210

\bibitem[1982]{taylor84}
Taylor, A. R., &amp; Gregory, P. C.
1984, ApJ, 283, 273

\bibitem[1966]{vdlaan66}
van der Laan, H.
1966, Nature, 211, 1131


%R 0001adass...0.....Z
%T Starlink Software Developments
%A Bly, Martin J.; Chipperfield, Alan J.; Giaretta, David L.; Graffagnino, Vito G.; McIlwrath, Brian K.; Platon, Roy T.; Wallace, Patrick T.; Allan, Alasdair; Berry, David S.; Davenhall, Clive; Draper, Peter W.; Gray, Norman; Jenness, Tim; Taylor, Mark B.
%J Astronomical Data Analysis Software and Systems XI, ASP Conference Proceedings, Vol. 281. Edited by David A. Bohlender, Daniel Durand, and Thomas H. Handley. ISBN: 1-58381-124-9. ISSN: 1080-7926. San Francisco: Astronomical Society of the Pacific, 2002, p. 513.
%D 00/2002
%F AA(Rutherford Appleton Laboratory, Didcot, Oxford, OX11 0QX, UK, <EMAIL>bly@star.rl.ac.uk</EMAIL>), AB(Rutherford Appleton Laboratory, Didcot, Oxford, OX11 0QX, UK), AC(Rutherford Appleton Laboratory, Didcot, Oxford, OX11 0QX, UK), AD(Rutherford Appleton Laboratory, Didcot, Oxford, OX11 0QX, UK), AE(Rutherford Appleton Laboratory, Didcot, Oxford, OX11 0QX, UK), AF(Rutherford Appleton Laboratory, Didcot, Oxford, OX11 0QX, UK), AG(Rutherford Appleton Laboratory, Didcot, Oxford, OX11 0QX, UK), AH(School of Physics, University of Exeter, Exeter, EX4 5EN, UK), AI(Centre for Astrophysics, University of Central Lancashire, Preston, PR1 2HE, UK), AJ(Institute for Astronomy, Royal Observatory, Blackford Hill, Edinburgh, EH9 3HJ, UK), AK(Dept. of Physics, University of Durham, Durham, DH1 3LE, UK), AL(Department of Physics and Astronomy, University of Glasgow, Glasgow G12 8QQ, Scotland, UK), AM(Joint Astronomy Centre, 660 N. A'ohoku Place, Hilo, HI 96720,USA), AN(Institute of Astronomy, Madingley Road, Cambridge, CB3 0HA, UK)

%K reduction: data, Starlink, ORAC, pipelines, virtual observatory

%B This paper describes Starlink's developments in several new areas, including
Grids, software management and distribution, data formats and GUI tools.

%Z \reference Allan, A., et al. 2001, \adassx, 459

\reference Allan, A., et al. 2002, \adassxi, 311

\reference Berry, D. S., 2001, Starlink User Note 183, Starlink Project, CCLRC

\reference Berry, D. S., &amp; Gledhill, T. M. 2001, Starlink User Note 223,
Starlink Project, CCLRC

\reference Bly, M. 1999, \adassviii, 451

\reference Currie, M. J., &amp; Berry, D. S. 2001, Starlink User Note 95,
Starlink Project, CCLRC

\reference Draper, P. W. 2000, \adassix, 615

\reference Economou, F., et al. 1999, \adassviii, 11

\reference Giaretta, D. L., et al. 2002, \adassxi, 20

\reference Jenness, T., et al. 2002, \adassxi, 243

\reference Mann, R., et al., 2002, \adassxi, 3

\reference McIlwrath, B. K., &amp; Giaretta, D. L., 2002, \adassxi, 475

\reference Shortridge, K., et al. 2001, Starlink User Note 86, Starlink
Project, CCLRC

\reference Warren-Smith, R. F., &amp; Berry, D. S. 2000, \adassix, 506

%%%%%%%%%%%%%%%
%% note that some of these references start with a dot, this is causing parser not to parse them, rendering them unrecognizable

%R 0000sdef....0.....Z
%T Absorption Studies with GRB Afterglows
%A Fiore, Fabrizio
%F Osservatorio Astronomico di Roma, via Frascati 33, Monteporzio I-00040, Italy
%J Scientific Drivers for ESO Future VLT/VLTI Instrumentation Proceedings of the ESO Workshop held in Garching, Germany, 11-15 June, 2001. p. 30.
%D 00/2002
%B GRB afterglows close to their peak intensity are among the brightest
sources in the sky. We discuss how GRB optical-to-X-ray afterglows can
be used as probes of the metal enrichment of the interstellar matter
in the GRB host galaxies and of the heating history of the
intergalactic matter in filaments along the line of sight.  The key
points of the proposed observation strategy are: 1) prompt GRB
observations (from minutes to a few hours); 2) high resolution
(R &cong; 10,000) OUV observations and 3) coordinated high resolution
(R &gt;&sim;1,000) X-ray observations.

%Z \bibitem Churchill, C. W., et al.: Astroph.\ Journal Suppl., 130, 91 (2000)

\bibitem
.Dav&egrave;, R. et al.: Astroph.\ Journal, 552, 473 (2001)

\bibitem
.Fiore, F., Nicastro, F., Savaglio, S., Stella, L. &amp; Vietri, M.:
Astroph.\ Journal Lett., 544, L7 (2000)

\bibitem
.Fiore, F. proceedings of the Yokohama symposium
`New Century of X-ray Astronomy', astro-ph/0107276 (2001)

\bibitem
.Frontera, F. et al.: Astroph.\ Journal Suppl., 127, 59 (2000)

\bibitem
.Galama, T. J., &amp; Wijers, A. M. J.: Astroph.\ Journal Lett., 549, L209
(2001)

\bibitem
.Gorosabel, J. et al.:  Astron.\ Astrophys, 339, 719 (1998)

\bibitem
.Kulkarni, S. R. et al.: Nature, 398, 389 (1999)

\bibitem
.in 't Zand, J. J. M. et al.: Astroph.\ Journal Lett., in press, astro-ph/0104362 (2001)

\bibitem
.Jha, S. et al.: Astroph.\ Journal Lett., 554, L155 (2001)

\bibitem
.Lee, J., et al.: Astroph.\ Journal Lett., 554, L13 (2001)

\bibitem
.Lu, L. et al.: Astroph.\ Journal Suppl., 107, 475 (1996)

\bibitem
.Metzger, M. R. et al.: Nature, 387, 878 (1997)

\bibitem
.Masetti, N. et al.: Astron.\ Astrophys, 374, 382 (2001)

\bibitem
.Perna, R. &amp; Loeb, A.: Astroph.\ Journal, 501, 467 (1998)

\bibitem
.Pettini, M., Ellison, S. L., Steidel, C. C. &amp; Bowen, D. V.:
Astroph.\ Journal,  510, 576 (1999)

\bibitem
.Pettini, M., Smith, L. J., King, D. L. &amp; Hunstead, W.:
Astroph.\ Journal, 486, 665 (1997)

\bibitem
.Piro, L. et al.: Science, 290, 955 (2000)


\bibitem
.Steidel, C. et al.: Astroph.\ Journal, 519, 1 (1999)

\bibitem
.Tripp, T. M., Savage, B. D., &amp; Jenkins, E. B.: Astroph.\ Journal Lett.,
534, L1 (2000)

\bibitem
.Vreeswijk, P. M. et al.: Astroph.\ Journal, 546, 672 (2001)




%R 0000luml....0.....Z
%J Lighthouses of the Universe: The Most Luminous Celestial Objects and Their Use for Cosmology Proceedings of the MPA/ESO/, p. 597
%D 00/2002
%T The Contribution of AGN to the Far Infrared Background
%A Risaliti, Guido; Elvis, Martin; Gilli, Roberto
%F AA(Harvard-Smithsonian Center for Astrophysics, 60 Garden Street, Cambridge, MA 02138, USA, and Osservatorio Astrofisico di Arcetri, Largo E. Fermi 5, I-50125, Firenze, Italy), AB(Osservatorio Astrofisico di Arcetri, Largo E. Fermi 5, I-50125, Firenze, Italy), AC(Osservatorio Astrofisico di Arcetri, Largo E. Fermi 5, I-50125, Firenze, Italy, and Johns Hopkins University,
Baltimore, MD, USA)
%B We use synthesis models for the X-ray background, together with
quasar Spectral Energy Distributions, to estimate
the contribution of Active Galactic Nuclei to the Far Infrared
Background and to the total luminosity of the Universe.
We find that the AGN contribution to the energy output of the
Universe is between 7% and 15%, and comparing this value with
the mass of supermassive black holes, we find that the average efficiency
of accretion in converting mass to energy must be at least 15%.
The contribution of AGN to the overall FIR background is on average low,
but it can be dominant at some wavelengths.

%Z \bibitem{risalitig:ref:a1} Elvis, M., et al., 1994, ApJS, 95, 1
\bibitem{risalitig:ref:a2} Fabian, A.C., &amp; Iwasawa, K., 1999, MNRAS 303, L34
\bibitem{risalitig:ref:a3} Gilli, R., Savati, M., &amp; Hasinger, G. 2001, A&amp;A, 366, 407
\bibitem{risalitig:ref:a4} Merritt, D., &amp; Ferrarese, L. 2001, MNRAS, 320, L30
\bibitem{risalitig:ref:a5} Salucci, p., Szuszkiewicz, E., Monaco, P., &amp; Danese, L. 1999, MNRAS, 307, 637
\bibitem{risalitig:ref:a6} Severgnini, P., et al. 2000, A&amp;A 360, 457




%R 0000osp.....0.....Z
%T Origin of Stars and Planets: The View from 2021
%A Trimble, Virginia
%F Dept. of Physics, Univ. of California, Irvine, CA 92697, USA, and Dept. of Astronomy, Univ. of Maryland, College Park, MD 20742, USA
%J The Origins of Stars and Planets: The VLT View. Proceedings of the ESO Workshop held in Garching, Germany, 24-27 April 2001, p. 493.
%D 00/2002
%B What (or who) made the Earth, and how? What made the Sun, and are there
others? Thus mankind has wondered for eons. Not surprisingly, the
answers have evolved a good deal over the decades, centuries, and
millenia. So, perhaps more importantly, have the meanings of the
questions. Here is an attempt to explore, ``as far as thought can
reach'', both back into the past and forward into the future, the
territory defined by such questions, illuminated by light (and
infrared) from the present workshop. If you decide to come along on
the quest, it is probably worth carrying in your baggage the related,
if strange-sounding, question, is star formation important?

%Z \bibitem{ambart60}
Ambartsumian, V. A. 1960. QJRAS 1, 152

\bibitem{badawy54}
Badawy, A. M. 1954. History of Egyptian Architecture (Gizah), p. 153

\bibitem{bahcall97}
Bahcall, J. N. &amp; Ostriker, J. P. (eds.) 1997. Unsolved Problems in
Astrophysics, Princeton Univ.\ press

\bibitem{brilliant85}
Brilliant, A. 1985. Pot Shots No.\ 3253, Brilliant Enterprises,
Santa Barbara

\bibitem{brush96}
Brush, S. G. 1996. A History of Modern Planetary Physics, Cambridge
Univ.\ Press

\bibitem{burbidge57}
Burbidge, E. M., Burbidge, G. R., Fowler, W. A. &amp; Hoyle, F. 1957.
RMP 29, 547

\bibitem{cox00}
Cox, A. N. (ed.) 2000. Allen's Astrophysical Quantities, Fourth
Edition, Springer

\bibitem{curtis21}
Curtis, H. D. 1921. Bull.\ NRC 2, 194

\bibitem{diamond98}
Diamond, J. 1998. Guns, Germs, and Steel, W. W. Norton

\bibitem{economist}
Economist, The 1999. Issue of 31 December

\bibitem{hoyle39}
Hoyle, F. &amp; Lyttleton, R. A. 1939. Proc.\ Cam.\ Phil.\ Soc.\ 35, 405,
595, 608

\bibitem{hubble24}
Hubble, E. P. 1924. NY Times, 23 November, p. 6

\bibitem{lang80}
Lang, K. R. 1980. Astrophysical Formulae, 2nd Ed., Springer

\bibitem{russell26}
Russell, H. N., Dugan, R. S., &amp; Stewart J. Q. 1926. Astronomy (Boston:
Ginn &amp; Co.)

\bibitem{shapley21}
Shapley, H. 1921. Bull.\ NRC 2, 171

\bibitem{shklov66}
Shklovskii, I. S., Sagan, C. 1966. Intelligent Life in the Universe,
Holden-Day

\bibitem{shu87}
Shu, F. H., Adams, F. C., &amp; Lizano, S. 1987. ARA&amp;A 25, 23

\bibitem{trimble75}
Trimble, V. 1975. RMP 47, 877

\bibitem{trimble97}
Trimble, V. 1997. In S. S. Holt &amp; L. G. Mundy (eds.) Star Formation
Near and Far, AIP Conf.\ Ser.\ 393, p. 15

\bibitem{trimble01}
Trimble, V. &amp; Aschwanden, M. 2001. PASP, 113 (Sept.\ issue)

%%%%%%%%%%%%%%%
%R 0000defi....0.....Z
%T Measuring Large-Scale Structure with the 2dF Galaxy Redshift Survey
%A Peacock, J. A.; The 2dF Galaxy Redshift Survey team; Colless, Matthew; Peacock, John; Baugh, Carlton M.; Bland-Hawthorn, Joss; Bridges, Terry; Cannon, Russell; Cole, Shaun; Collins, Chris; Couch, Warrick; Cross, Nicholas; Dalton, Gavin; Deeley, Kathryn; De Propris, Roberto; Driver, Simon; Efstathiou, George; Ellis, Richard S.; Frenk, Carlos S.; Glazebrook, Karl; Jackson, Carole; Lahav, Ofer; Lewis, Ian; Lumsden, Stuart; Maddox, Steve; Madgwick, Darren; Norberg, Peder; Percival, Will; Peterson, Bruce; Sutherland, Will; Taylor, Keith
%F AA(Institute for Astronomy, University of Edinburgh, Royal Observatory, Edinburgh EH9 3HJ, UK), AC(ANU), AD(ROE), AE(Durham), AF(AAO), AG(AAO), AH(AAO), AI(Durham), AJ(LJMU), AK(UNSW), AL(St Andrews), AM(Oxford), AN(UNSW), AO(UNSW), AP(St Andrews), AQ(IoA), AR(Caltech), AS(Durham), AT(JHU), AU(ANU), AV(IoA), AW(AAO), AX(Leeds), AY(Nottingham), AZ(IoA), BA(Durham), BB(ROE), BC(ANU), BD(ROE), BE(Caltech)
%J Deep Fields, Proceedings of the ESO/ECF/STScI Workshop held at Garching, Germany, 9-12 October 2000. Stefano Cristiani, Alvio Renzini, Robert E. Williams (eds.). Springer, 2001, p. 221.
%D 12/2001

%B The 2dF Galaxy Redshift Survey is the first to measure more than
100,000 redshifts. This allows precise measurements of many of the key
statistical measures of galaxy clustering, in particular
redshift-space distortions and the large-scale power spectrum.  This
paper presents preliminary 2dFGRS results in these areas.

%Z \bibitem{} Ballinger W. E., Peacock J. A., Heavens A. F., 1996, MNRAS, 282, 877
\bibitem{} Baugh C. M., Efstathiou G., 1994, MNRAS, 267, 323
\bibitem{} Benoist C., Maurogordato S., da Costa L. N., Cappi A., Schaeffer R., 1996, ApJ, 472, 452
\bibitem{} Carlberg R. G., Yee H. K. C., Morris S. L., Lin H., Hall P. B., Patton D., Sawicki M., Shepherd C. W., 2000, ApJ, 542, 57
\bibitem{} Davis M., Peebles, P. J. E., 1983, ApJ, 267, 465
\bibitem{} Folkes S. J. et al., 1999, MNRAS, 308, 459
\bibitem{} Feldman H. A., Kaiser N., Peacock J. A., 1994, ApJ, 426, 23
\bibitem{} Freedman W. L. et al., 2000, astro-ph/0012376
\bibitem{} Jaffe A. et al., 2000, astro-ph/0007333
\bibitem{} Kaiser N., 1987, MNRAS, 227, 1
\bibitem{} Lewis I., Taylor K., Cannon R. D., Glazebrook K., Bailey J. A., Farrell T. J., Lankshear A., Shortridge K., Smith G. A., Gray P. M., Barton J. R., McCowage C., Parry I. R., Stevenson J., Waller L. G., Whittard J. D., Wilcox J. K., Willis K. C., 2000, MNRAS, submitted
\bibitem{} Loveday J., Maddox S. J., Efstathiou G., Peterson B. A., 1995, ApJ, 442, 457
\bibitem{} Maddox S. J., Efstathiou G., Sutherland W. J., Loveday J., 1990a, MNRAS, 242, 43p
\bibitem{} Maddox S. J., Sutherland W. J., Efstathiou G., Loveday J., 1990b, MNRAS, 243, 692
\bibitem{} Maddox S. J., Efstathiou G., Sutherland W. J., 1990c, MNRAS, 246, 433
\bibitem{} Meiksin A. A., White M., Peacock J. A., 1999, MNRAS, 304, 851
\bibitem{} Mould J. R. et al., 2000, ApJ, 529, 786
\bibitem{} Peacock J. A., Dodds S. J., 1996, MNRAS, 280, L19
\bibitem{} Peacock J. A. et al., 2000, Nature, submitted
\bibitem{} Schlegel D. J., Finkbeiner D. P., Davis M., 1998, ApJ, 500, 525
\bibitem{} Verde L., Heavens A. F., Matarrese S., Moscardini L., 1998, MNRAS, 300 747


%R 0001defi....0.....Z
%T FIR Emission from Dust in Abell Clusters
%A Stickel, M.; Klaas, U.; Lemke, D.; Mattila, K.
%F AA(MPI f&uuml;r Astronomie, K&ouml;nigstuhl 17, D--69117 Heidelberg, Germany), AB(MPI f&uuml;r Astronomie, K&ouml;nigstuhl 17, D--69117 Heidelberg, Germany), AC(MPI f&uuml;r Astronomie, K&ouml;nigstuhl 17, D--69117 Heidelberg, Germany), AD(Helsinki University Observatory, P. O.Box 14, SF--00014 Helsinki, Finland)
%J Deep Fields, Proceedings of the ESO/ECF/STScI Workshop held at Garching, Germany, 9-12 October 2000. Stefano Cristiani, Alvio Renzini, Robert E. Williams (eds.). Springer, 2001, p. 216.
%D 12/2001

%B The ISOPHOT instrument aboard ISO has been used to search for extended
FIR emission from intracluster dust in six Abell clusters.  Except for
Abell 1656, no clear evidence for a systematic trend in the
I<SUB>120</SUB> microns / I<SUB>180</SUB> microns surface brightness ratios was
found. Any existing intracluster dust is weak (&lt; 0.1 MJy sr<SUP>-1</SUP>)
or has properties quite similar to the foreground
galactic cirrus, in which case it will not show up in the FIR color profiles.

%Z bibitem {Dweketal90} Dwek E., Rephaeli Y., Mather J. 1990, ApJ 350, 104

bibitem {Maoz95} Maoz D., 1995, ApJ 455, L115

bibitem {Popescuetal00} Popescu C. C., Tuffs R. J., Fischera J., V&ouml;lk H. 2000, A&amp;A 354, 480

bibitem {Stickeletal98} Stickel M., Lemke D., Mattila K., Haikala L. K., Haas M. 1998, A&amp;A 329, 55

bibitem {Stickeletal01} Stickel M., Klaas U., Lemke D., Mattila K., 2001, A&amp;A, to be submitted

bibitem {Wiseetal93} Wise M. W., O'Connell R. W., Bregman J. N., Roberts M. S. 1993, ApJ 405, 94



%R 0000IAUS....0.....Z
%T Adaptive Optics Imaging of Faint Companions: Current &amp; Future Prospects
%A Close, Laird M.
%F ESO, Garching, Germany
%J Birth and Evolution of Binary Stars, Proceedings of IAU Symposium No. 200 on The Formation of Binary Stars, held 10-15 April, 2000, in Potsdam, Germany. Edited by Bo Reipurth and Hans Zinnecker, 2001, p. 555.
%D 00/2001
%B I briefly describe how diffraction-limited imaging with adaptive
optics (AO) can detect low mass companions (young massive brown dwarfs
for example). I review how current curvature AO systems can already
detect point sources 1 million times fainter at separations of 3
arcsec in median seeing (0.65 arcsec). I show real examples of very
faint companion detections made with the University of Hawaii AO
system located at CFHT on Mauna Kea around the young (2 Myr) nearby
(132 pc) Herbig Ae/Be star MWC480.  Moreover, I show that the four
faint (H=18--19 mag) companions within 6 arcsec of MWC480 (H=7.0 mag)
are unlikely to be physical since they are non-common proper motion
objects. I point out that the current 8--10m class AO systems will
detect even fainter companions at closer separations with 0.03--0.06
arcsec NIR imaging.

%Z \reference Burrows, A., et al. 1997, ApJ, 491, 856
\reference Graves, J. E. et al. 1998, Proc SPIE 3353
--- http://www.ifa.hawaii.edu/ao/publi.html
\reference Close, L. M, et al. 1998, ESO/OSA AAO Proc., 109
--- http://www.ifa.hawaii.edu/ao/publi.html
\reference Close, L. M. 2000, Proc. SPIE 4007, 758
(a full review of AO past &amp; future)




%R 0000cksa....0.....Z
%T A Near-Infrared Imaging Survey of Nearby Spiral Galaxies
%A Stedman, S.; Knapen, J. H.; Bramich, D. M.
%F AA(University of Hertfordshire, Dept of Physical Sciences, Hatfield, Herts. AL10 9AB, UK)
%J The Central Kiloparsec of Starbursts and AGN: The La Palma Connection, ASP Conference Proceedings Vol. 249. Edited by J. H. Knapen, J. E. Beckman, I. Shlosman, and T. J. Mahoney. ISBN: 1-58381-089-7. San Francisco: Astronomical Society of the Pacific, 2001, p. 187.
%D 00/2001
%B K-band imaging was carried out on a sample of 57 spiral galaxies to map
their stellar backbones, examine the underlying mass distribution across
the arm and interarm regions, and perform bulge:disk decomposition.  As
within the broader aims of the project, which also use B, I, and
H&alpha; images, our results will be examined in the light of global
parameters including Hubble stage, Elmegreen arm class, and quantity and
distribution of recent massive star formation and dust.

%Z \reference Beckman, J. E., Peletier, R. F., Knapen, J. H., Corradi,
R. L. M., &amp; Gentet, L. J. 1996, \apj, 467, 175

\reference de Vaucouleurs, G., de Vaucouleurs, A., &amp; Corwin, H. G., Jr. 1976,
Second Reference Catalogue of Bright Galaxies (Austin: Univ.
Texas Press)

\reference de Vaucouleurs, G., de Vaucouleurs, A., Corwin, H. G., Buta, R. J.,
Paturel, G., &amp; Fouqu&eacute;, P. 1991, Third Reference Catalogue of
Bright Galaxies (New York: Springer)

\reference Elmegreen, D. M., &amp; Elmegreen, B. G. 1987, \apj, 314, 3

\reference Elmegreen, D. M, Chromey, F. R., Sawyer, J. E., &amp; Reinfeld,
E. L. 1999, \aj, 118, 777

\reference Grosbol, P. J., &amp; Patsis, P. A. 1998, \aap, 336,
840

\reference Knapen, J. H., &amp; Beckman, J. E. 1996,\mnras, 283, 251

\reference P&eacute;rez-Ram&iacute;rez, D., Knapen, J. H., Peletier, R. F.,
Laine, S., Doyon, R., &amp; Nadeau, D. 2000, \mnras, 317, 234

\reference Regan, M. W., &amp; Elmegreen, D. M. 1997, \aj, 114, 965

\reference Rix, H.-W. 1993, \pasp, 105, 999

\reference Rix, H.-W., &amp; Zaritsky, D. 1995, \apj, 447, 82

\reference Seigar, M. S., &amp; James, P. A. 1998, \mnras, 299, 685

\reference Stedman, S., &amp; Knapen, J. H. 2001, \apss, in press




%R 0000qarr....0.....Z
%Z \bibitem{}
Cagnoni, I., Della Ceca, R. and Maccacaro T., 1998, Astrophys. J. 493,
54--61

\bibitem{}
Fr&ouml;hlich, C., 1988, ESA SP-418, 7--10

\bibitem{}
Gabriel, A.H., Turck-Chieze, S., Garcia, R.A., Pall&eacute;, P.L.,
Boumier, P., Thi&eacute;ry, S., Grec, G., Ulrich, R.K., Bertello, L.,
Roca-Cort&egrave;s, T. and Robillot, J.M., 1998, ESA SP-418, 61--66

\bibitem{}
Gehrels, N., Macomb, D.J., Bertsch, D.L., Thomson, D.J. and Hartman,
R.C., 2000, Nature 404, 363--365

\bibitem{}
Giacconi, R., Gursky, H., Paolini, F. and Rossi, B., 1962,
Phys. Rev. Letters 9, 439--443

\bibitem{}
Gilmore, G., de Boer, K., Favata, F., Hoeg, E., Lattanzi, M.,
Lindegren, L., Lur&eacute;, X., Mignard, F., Perryman, M. and de Zeeuw,
P.T., 2000, Proceedings of SPIE 4013, in press, March 2000, Munich,
J.B. Brekinridge and P. Jakobsen editors

\bibitem{}
Hasinger, G., 2000, Nature 404, 443--446

\bibitem{}
Jansen, F.A., 1999, ESA Bulletin 100, 9--12

\bibitem{}
Mendis, D.A., 1987, Astron. and Astrophys. 187, no. 2, 939--948




%R 0000pao.....0.....Z
%T Further Detections of OH Masers in Carbon Stars with Silicate Features
%A Szymczak, M.; Szczerba, R.; Chen, P. S.
%F AA(Toru&nacute; Centre for Astronomy, Nicolaus Copernicus University, Toru&nacute;, Poland), AB(Nicolaus Copernicus Astronomical Center, PAS, Toru&nacute;, Poland), AC(Yunnan Astronomical Observatory, CAS, Kunming, P. R. China)
%J Post-AGB Objects as a Phase of Stellar Evolution,
   Proceedings of the Toru&nacute; Workshop held July 5-7, 2000.
   Edited by R. Szczerba and S. K. G&oacute;rny.
   Publisher: Kluwer Academic Publishers, Boston/Dordrecht/London, 2001.
%D 09/2001
%B A sample of J--type carbon stars was searched for OH maser emission. The new
detection of three OH lines towards two silicate carbon
stars is reported. In V778 Cyg, previously known as the main--lines (1665
and 1667 MHz) maser source, the satellite 1612 MHz emission was discovered
while in NSV 2814 the main OH lines were detected. The presence of OH maser
lines confirms the former suggestion that oxygen--rich material is located in
the vicinity (&cong; 10<SUP>15-16</SUP> cm) of silicate carbon stars.

%Z \bibitem[1991]{barnbaum91} Barnbaum C., Morris M., Likkel L, et al.,
              1991, A&amp;A 251, 79

\bibitem[1999]{chen99} Chen P.S., Wang X.H., Wang F., 1999,
              Chin. Astron. Astrophys. 23, 371

\bibitem[1994]{engels94} Engels D., 1994, A&amp;A 285, 497

\bibitem[1994]{engelsleinert94} Engels D., Leinert Ch., 1994, A&amp;A 282, 858

\bibitem[1999]{jura99} Jura M., Kahane C., 1999, ApJ 521, 302

\bibitem[1990]{lambert90} Lambert D.L., Hinkle K.H., Smith V.V.,
               1990, AJ 99, 1612

\bibitem[1986]{littlemarenin86} Little--Marenin I.R., 1986, ApJ 307, L15

\bibitem[1988]{littlemarenin88} Little--Marenin I.R., Benson P.J., Dickinson
              D.F., 1988, ApJ 330, 828

\bibitem[1994]{littlemarenin94} Little--Marenin I.R., Sahai R., Wannier P.G.,
              et al., 1994, A&amp;A 281, 451

\bibitem[1990]{lloydevans90} Lloyd Evans T., 1990, MNRAS 243, 336

\bibitem[1999]{ohnaka99} Ohnaka K., Tsuji T., 1999, A&amp;A 345, 233

\bibitem[1989]{stephenson89} Stephenson C.B., 1989, Pub. Warner and Swasey
               Obs., Vol. 3, No.2 (CCGCS)

\bibitem[1991]{telintelhekkert91} te Lintel Hekkert P., Caswell J.L.,
               Habing H.J., et al., 1991, A&amp;AS 90, 327

\bibitem[1999]{trams99} Trams N.R., van Loon J.Th., Zijlstra A.A., et al.,
              1999, A&amp;A 344, L17

\bibitem[1986]{willems86} Willems F.J., de Jong T., 1986, ApJ 309, L39

\bibitem[2000]{yamamura00} Yamamura I., Dominik C., de Jong T., et al.,
               2000, A&amp;A 363, 629


%R 0000gge.....0.....Z
%T Constraints on High-Velocity Cloud
Analogs in External Groups and Galaxies
%A Zwaan, M.
%D 00/2001
%J Gas and Galaxy Evolution, ASP Conference Proceedings, Vol. 240. Edited by John E. Hibbard, Michael Rupen, and Jacqueline H. van Gorkom. San Francisco: Astronomical Society of the Pacific, ISBN: 1-58381-077-3, 2001, p. 523
%B The hypothesis that high-velocity clouds (HVCs) are Local Group
satellites at typical distances of a few hundred kpc to 1.5 Mpc, and
with HI masses &sim; 10<SUP>7</SUP> M<SUB>&odot;</SUB>, is highly inconsistent with
(1) the results from blind HI surveys, (2) targeted surveys
for intra-group HI clouds, and (3) QSO absorption line
statistics.
%Z
\reference Blitz, L., Spergel, D.N., Teuben,
   P.J., Hartmann, D., \& Burton, W.B. 1999, \apj, 514, 818 [BSTHB]
\reference Braun, R., \& Burton, W.B 1999, \aap, 341, 437 [BB]
\reference Braun, R., \& Burton, W.B.  2000, \aap, 354, 853
\reference Charlton, J.C., Churchill, C.W., \& Rigby, J.R. 2000, \apj,
  544, 702
\reference Hu, E.M., Kim, T., Cowie,  L.L., Songaila, A. \& Rauch,
        M. 1995, \aj, 110, 1526
\reference Kamionkowski, M. \& Liddle, A.R. 2000, \prl, 84, 4525
\reference Kilborn, V. et al. 2000, \aj, 120, 1342
\reference Klypin, A.A., Kravtsov, A.V.,
  Valenzuela, O., \& Prada, F. 1999, \apj, 522, 82
\reference Lo, K.Y. \& Sargent, W.L.W. 1979, \apj, 227, 756
\reference Moore, B., Ghigna, F., Governato, F., Lake, G., Stadel J.,
    \&  Tozzi, P. 1999, \apj, 524, L19
\reference Petitjean, P., Webb, J.K., Rauch, M., Carswell, R.F. \&
        Lanzetta, K. 1993, \mnras, 262, 499
\reference Putman, M.E.  et al. 1998, Nature, 394, 752
\reference Rao, S.M. \& Turnshek, D.A. 2000, \apjs, 130, 1
\reference Spergel, D.N. \& Steinhardt, P.J. 2000, \prl, 84, 3760
\reference Stengler-Larrea, E.A., et al. 1995, \apj, 444, 64
\reference Wakker, B.P., \& van Woerden, H. 1991, \aap, 250, 509
\reference Zwaan, M.A. 2000, Ph.D. thesis, University of Groningen
\reference Zwaan, M.A. \& Briggs, F.H. 2000, \apj, 530, L61
\reference Zwaan, M.A., Briggs, F.H.,
  Sprayberry, D., \& Sorar, E. 1997, \apj, 490, 173



%R 0000bhbg....0.....Z
%Z \bibitem{WL1} Wellstein, S., & Langer, N 1999, submitted to A&A, astro-ph/9904256

\bibitem{WL2} Wellstein, S., & Langer, N 2000, this issue

\bibitem{Fr1} Fryer, C. L. 1999, ApJ, 522, 413

\bibitem{Woo1} Woosley, S. E., & Weaver, T. A. 1995, ApJS, 101, 181

\bibitem{MW1} MacFadyen, A., & Woosley, S. E. 1999, accepted by ApJ, astro-ph/9810274

\bibitem{LH1} Langer, N., & Henkel, C., 1995, Space Science Reviews, 74, 343

\bibitem{FWH1} Fryer, C. L., Woosley, S. E., & Hartmann, D., accepted by ApJ, astro-ph/9904122

\bibitem{PFH1} Pei, Y. C., Fall, S. M., & Hauser, M. G. 1999, ApJ, 522, 604

\bibitem{KG1} Kalogera, V., & Webbink, R. 1998, ApJ, 493, 351



%R 0000fepc....0.....Z
%T Direct Detection of Hot Extrasolar Planets Using Differential Interferometry
%A Lopez, B.; Petrov, R. G.
%F AA(Observatoire de la C&ocirc;te d'Azur, UMR 6528, BP 4229, F-06304 Nice cedex 4, France), AB(UMR 6525 ``Astrophysique'' du CNRS et de l'Universit&eacute; de Nice Sophia Antipolis, Parc Valrose, F-06108 Nice cedex 2, France)
%J From Extrasolar Planets to Cosmology: The VLT Opening Symposium, Proceedings of the ESO Symposium held at Antofagasta, Chile, 1-4 March 1999. Edited by Jacqueline Bergeron and Alvio Renzini. Berlin: Springer-Verlag, 2000. p. 565.
%D 00/2000
%B We evaluate the potential of differential interferometry with the VLT
or Keck interferometers for the direct detection and the study of
atmospheric characteristics of hot giant extrasolar planets around
nearby stars.

Differential interferometry has been shown to allow unbiassed
measurements of phase and/or visibility variations with wavelength.
For a star and planet system like for any binary, these measurements
at several orbital phases yield the angular separation and the
spectrum of the components.

We present an evaluation of the fundamental measurement uncertainty
resulting from source and sky photon noise and detector noise.  It
shows that, with two 8 meters telescopes and a 40 meters baseline, we
should detect Jupiter-like planets up to 0.1 or 0.2 astronomical units
away from a solar type star at 10 pc.

%Z \bibitem{Bur} Burrows A., Marley M.,
Hubbard W. B. et al 1997, ApJ 491, 856

\bibitem{Bracewell} Bracewell R.N. 1978, Nature 274, 780

\bibitem{Chelli} Chelli A. and Petrov R.G. 1995, A&amp;AS 109, 401

\bibitem{Vincent} Coud&eacute; du Foresto V. 1999,
this symposium (Parallel Workshop 2)

\bibitem{Glinde} Glindeman A., this symposium (Parallel Workshop 2)

\bibitem{Jean-Marie} Mariotti J.-M., Coud&eacute; du Foresto V.,
Perrin G. 1996, AAS DPS meeting 28, 1207

\bibitem{lag} Lagarde, S., 1994, Ph. D. Thesis, University of Nice, France

\bibitem{Leinert} Leinert C., Graser U. et al
1998, report of the MIDI VLTI instrument
for the ESO concept review of October 1998

\bibitem{Menten}  Menten K.M., this symposium (Plenary sessions)

\bibitem{Romain0} Petrov, R.G., et all, 1991, ESO-NOAO conference on High Resolution Imaging By
Interferometry II, Garching
bei M&uuml;nchen, Germany

\bibitem{Romain} Petrov R.G. et al 1999, report of the
AMBER VLTI instrument
for the ESO concept review of January 1999

\bibitem{Quirrenbach} Quirrenbach A. 1999,
this symposium (Parallel Workshop 2)

\bibitem{Van Belle} Van Belle G. 1999, this symposium (Parallel Workshop 2)



%R 0000fist....0.....Z
%T Light and Metals from Population III Stars
%A Tumlinson, Jason
%F CASA, University of Colorado, Boulder, CO, 80309, USA
%J The First Stars. Proceedings of the MPA/ESO Workshop held at Garching, Germany, 4-6 August 1999. Achim Weiss, Tom G. Abel, Vanessa Hill (eds.). Springer.
%D 00/2000
%B We present zero-age models of Population III stars that demonstrate
these stars have enhanced ionizing photon production relative to
metal-enriched stars. We comment on the possibility that these stars
reionized the universe and calculate a critical rate of metal-free
star formation required to reionize H by z = 5. We speculate on
the effects of our models on various other cosmological processes.

%Z \bibitem{tjTL} Hubeny, I., Lanz, T. 1995, ApJ, 439, 875
\bibitem{tjS99} Leitherer, C., et al. 1999, ApJS, 123, 3
\bibitem{tjMad} Madau, P., Mozetti, L., & Dickinson, M. 1998, ApJ, 498, 106
\bibitem{tjTum1} Tumlinson, J., & Shull, J. M. 1999, ApJL, submitted
\bibitem{tjZheng} Zheng, W., et al. 1997, ApJ, 475, 469
\bibitem{tjww} Woosley, S. E., & Weaver, T. A. 1995, ApJS, 101, 181


%%%%%%%%%%%%%%%%%%%%%%%%%%%%%%%%%%%%%%%%%%%%%%%%%%%%%%%%%%%%%%%

\adsbibcode{0000dgeu....0.....Z}
%%PAGE 315

\title{Globular Cluster Formation in Mergers}
\author{Bradley Whitmore}
\affil{Space Telescope Science Institute\\
           3700 San Martin Dr., Baltimore, MD 21218}

\begin{abstract}

Until recently, it was believed that essentially all globular clusters
were formed very early in the history of the universe, and the
investigation of their formation was primarily a theoretical
question. However, the recent discovery of young compact star clusters
formed during the merging of two galaxies provides the chance to
directly observe the formation and evolution of what appear to be
young globular clusters. These new observations have revitalized the
study of globular clusters, and are providing important new insights
into the merger process itself.

\end{abstract}

\keywords{young globular clusters, interacting galaxies}

\begin{references}

\reference Ashman, K. M., \& Zepf, S. E. 1992, \apj, 384, 50
\reference  Barth, A. J., Ho, L. C., Filippenko, A. V., \& Sargent, W.L.W. 1995, AJ, 110, 1009
\reference  Borne, K. D. et al. 1996, in Science with the Hubble Space Telescope - II, eds. P. Benvenuti, F. D. Macchetto, \& E. J. Schreier, 239.
\reference Brodie, J. P., Schroder, L. L., Huchra, J. P., Phillips, A. C., Kissler-Patig, M., \& Forbes, D. F. 1998, \aj, 116, 691
\reference  Bruzual A. G., \& Charlot, S. 1996, in preparation
\reference  Burstein, D. 1987, in Nearly Normal Galaxies, ed. S. Faber (Springer, New York) p. 47
\reference  Carlson, M. N. \& the WFPC2 team 1998, \aj, 115, 1778
\reference  Crabtree, D. R. \& Smecker-Hane, T.A. 1994, BAAS, 26, 1494
\reference  Hilker, M., \& M. Kissler-Patig 1996, A\&A, 314, 357
\reference  Ho, L. C. \& Filippenko, A. V. 1996, ApJ, 472, 600
\reference  Ho, L. C.  1996, in Starburst Activity in Galaxies, ed. J. Franco, R. Terlevich, \& G. Tenorio-Tagle
\reference  Holtzman, J. A. et al. (the WFPC2 team) 1996, AJ, 112, 416
\reference  Holtzman, J. A. et al. (the WFPC team) 1992, AJ, 103, 691
\reference  Johnson, K. E., Vacca, W. D., Leitherer, C., Conti, P., \& Lipsey, S. J. 1998, \aj, 117, 1708
% \reference  Kennicutt, R. C. \& Chu. Y.-H. 1988, AJ, 95, 720
% \reference  Kundu, A., Whitmore, B. C., Sparks, W. B., Macchetto, F. D., Zepf, S. E., \& Ashman, K., 1998, AJ, submitted
\reference Larsen, S. S. \& Richtler, T. 1999, A\&A, 345, 59
\reference  Lee, M. G., Kim, E., \& Geisler, D. 1997, AJ, 114, 1824
\reference  Lutz, D. 1991, A\&A, 245, 31
\reference  Meurer, G. R., Heckman, T. M., Leitherer, C., Kinney, A., Robert, C.,  \& Garnett, D. R. 1995, AJ, 110, 2665
\reference Mihos, J. C.,  Bothun, G. D., \& Richstone D. O. 1993, ApJ, 418, 82
\reference   Miller, B. W., Whitmore, B. C., Schweizer, F., \& Fall, S. M. 1997, AJ, 114, 2381
\reference  Richer, H. B., Crabtree, D. R., Fabian, A. C., \& Lin, D. N. C. 1993, AJ, 105, 877
\reference  Sanders, D. B. et al. 1988, ApJ, 325, 74
\reference  Schweizer, F.  1982, ApJ, 252, 455
\reference  Schweizer, F.  1996,  in The Nature of Elliptical Galaxies, Proceedings of the Second Stromlo Symposium, ed. M. Arnaboldi, G. S. Da Costa, \& P. Saha
\reference  Schweizer, F., Miller, B.,  Whitmore, B. C., \& Fall, S. M. 1996,
AJ, 112, 1839
\reference  Schweizer, F., \& Seitzer, P. 1998, AJ , 116, 2206
\reference  Schweizer, F. 1987, in Nearly Normal Galaxies, ed. S. Faber (Springer, New York) p. 18
\reference  Stiavelli, M., Panagia, N., Carollo, M., Romaniello, M., Heyer, I., \& Gonzaga, S. 1998, ApJL, 492, L135
\reference  Surace, J. A. et al. 1997, ApJ, in press
\reference  Toomre, A. 1977, in The Evolution of Galaxies and Stellar Populations, ed. B. M. Tinsley \& R. B. Larson (Yale, New Haven), p. 401
\reference  Tyson J. A. et al. 1998, \aj, 116, 102
\reference  van den Bergh, S. 1995, Nature, 374, 215
\reference  van den Bergh, S. 1990, in Dynamics and Interactions of Galaxies, edited by R. Wielen (Springer, Heidelberg), p. 492
\reference  Whitmore, B. C. 1999 in Galaxy Interactions at Low and High Redshift, edited by J.E Barnes \& D.B. Sanders, (IAU, Netherlands), 251
\reference  Whitmore, B. C., \& Schweizer, F. 1995, AJ, 109, 960
\reference  Whitmore, B. C., Miller, B. W., Schweizer, F., \& Fall, S. M. 1997, AJ, 114, 1797
\reference  Whitmore, B. C., Schweizer, F., Leitherer, C., Borne, K., \& Robert, C.  1993, AJ, 106, 1354
\reference  Whitmore, B. C., Zhang, Q., Leitherer, C., Fall, S. M., Schweizer, F.,  \& Miller, B. W.  1999, \aj, 118, 1551
\reference  Zepf, S. E., \& Ashman, K. M. 1993,  MNRAS, 264, 611
\reference  Zepf, S.E., Ashman, K.M., English, J., Freeman, K.C., \& Sharples, R.M. 1999, AJ, 118, 752


\end{references}


------------------------

\adsbibcode{0000obco....0.....Z}
\title{Ly$\alpha$  forest and the total
 absorption cross-section of
galaxies -- an example of  the NTT SUSI Deep Field}

\author{Srdjan Samurovi\'c\altaffilmark{1}}
\affil{Public Observatory, Gornji Grad 16, 11000 Belgrade, SERBIA}
\author{Milan M.~\'{C}irkovi\'{c}\altaffilmark{2}}
\affil{Dept. of Physics \& Astronomy,
SUNY at Stony Brook,
Stony Brook, NY 11794-3800, USA}
\affil{Astronomical Observatory, Volgina 7, 11000 Belgrade, SERBIA}

\begin{abstract}
By extrapolating the accumulated low-redshift data on the absorption radius of galaxies
and its luminosity scaling, it is possible to predict the total absorption cross-section
of the gas associated with collapsed structures in the universe at any given epoch. This
prediction can be verified observationally through comparison with the well-known spatial
distribution of the QSO absorption systems. In this way, it is shown that HDF, NTT SUSI
Deep Field and other such data give further evidence for the plausibility of origin of
the significant fraction of the Ly$\alpha$ forest in haloes of normal galaxies.
\end{abstract}

\begin{references}


\reference{Arnouts, S., D'Odorico, S., Cristiani, S.,
 Fontana, A., Giallongo. E.
and Zaggia, S. 1998, {A\&A}, in preparation }


\reference{Bertin, E. and Arnouts, S. 1996, {A\& AS},
{117}, 393}


%\reference{Bertin, E.: 1996, {\it SExtractor 1.0a User's Guide}.}


\reference{Chen, H.-W., Lanzetta, K. M., Webb, J. K.
and Barcons, X. 1998, {ApJ}, {498}, 77}

\reference{Cristiani, S., D'Odorico, S., D'Odorico, V., Fontana, A., Giallongo, E.
and Savaglio, S.   1997, {MNRAS}, {285},
209}

\reference{\'Cirkovi\'c, M.M., Samurovi\'c, S., Ili\'c, D.
 and Petrovi\'c, J. 1997, {Bull. Astron. Belgrade}, {156}, 37}



\reference{Fernandez-Soto, A.,
Lanzetta, K.M., Yahil, A. and  Chen, H.-W.
1997,      in {Proceedings of the XIII IAP
       Colloquium,} P. Petitjean and S. Charlot,
    Paris: Editions Fronti\`eres,
        402}




\reference{Giallongo, E., D'Odorico, S.,
Fontana, A., Cristiani, S., Egami, E.,
Hu, E. and McMahon, R.G. 1998, {AJ}, {115}, 2169}

\reference{Kim, T.-S., Hu, E. M.,
Cowie, L. L. and Songaila, A. 1997, {AJ}, {114}, 1}


\reference{Lanzetta, K.M.,  Bowen, D.V.,
  Tytler, D. and Webb, J.K. 1995,
{ApJ},  {442}, 538}




\reference{Lanzetta, K. M., Yahil, A. and
Fernandez-Soto, A. 1996, {Nature},
{386}, 759}

\reference{Schechter, P. 1976, {ApJ}, {203}, 297 }



\reference{Shanks, T., Metcalfe, N., Fong, R., McCracken, H.J.,
Campos, A. and Gardner, J.P. 1998, astro-ph/9801296.}

\reference{Steidel, C.C.,  Dickinson, M. and Persson, S.E.
 1994, {ApJ}, {437}, L75}


\reference{Yahil, A., Lanzetta, K.M. and   Fernandez-Soto, A. 1998,
       in {Proceedings of the 12th Potsdam Cosmology
       Workshop}, V.~M\"uller, S.~Gottl\"ober, J.~P.~M\"u\-cket,
J.~Wambsganss, Singapore:~World Scientific, 1}

\reference{Willmer, C.N.A.    1997, {AJ}, {114}, 898}

\end{references}

----------------------

\adsbibcode{0000gady....0.....Z}
\setcounter{page}{507}
\title{The Identification of Ly$\alpha$ Absorbers at Low Redshift}
\author{Suzanne M. Linder}
\affil{Pennsylvania State University}

No abstract

\begin{references}
\reference Bowen, D. V., Blades, J. C., \& Pettini, 1996, \apj, 464, 141
\reference Chen, H. -W., Lanzetta, K. M., Webb, J. K., \& Barcons, X. 1998,
\apj, 498, 77
\reference Linder, S. M. 1998a, \apj, 495, 637
\reference Linder, S. M. 1998b, 171st IAU Proceedings (in press,
astro-ph/9810162)
\reference Tripp, T. M., Lu, L., \& Savage, B. D. 1998, \apj, 508, in press
\reference
\end{references}

\adsbibcode{0000cpg.....0.....Z}
File: sofue.tex
Page:		 \setcounter{page}{514}
\begin{references}

\reference  Anantharamaiah, K. R., Pedlar, A., Ekers, R. D., and
Goss, W. M. 1991, MNRAS, 249, 262.

\reference {Lang, C. 1999, this volume, p.~\pageref{clang}.}

\reference {Lesch, H. and Reich, W. 1992, AA, 264, 493.}

\reference Morris, M. 1996, in {\it Unsolved Problems of the Milky Way,
Proc. IAU Symp. 169},
eds. L. Blitz, P. Teuben, (Kluwer Academic Publishers, Dordrecht) p.247.

\reference Reich, W. 1989, in {\it The Center of the Galaxy,
Proc. IAU Symp. 136},
ed. M. Morris, (Kluwer Academic Publishers, Dordrecht) p.265.

\reference  Reich, W.,  Sofue, Y., Wielebinski, R. and Seiradakis, J. H.
1988, AA, 191, 303.

\reference  Serabyn, E., and G\"usten, R. 1991, AA, {\bf 242}, 376.



\reference {Sofue, Y., Inoue, M., Handa, T., Tsuboi, M., Hirabayashi, H.,
 Morimoto, M., Akabane, K. 1986, PASJ,  38, 483}

\reference {Sofue, Y., Murata, Y., and Reich, W. 1992, PASJ, 44, 367}

\reference {Sofue, Y., Reich, W., Inoue, M., Seiradakis, J. H. 1987,
PASJ,  39, 359.}

\reference {Tsuboi, M., Inoue, M., Handa, T., Tabara, H., Kato, T., Sofue, Y.,
and
Kaifu, N. 1986, AJ, 92, 818.}

\reference {Yusef-Zadeh, F., Morris, M., Chance, D. 1984, Nature, 310, 557.}


\reference {Yusef-Zadeh, F., 1986, PhD Thesis, Columbia University.}


\end{references}
------------------------------------------------------------------------
\adsbibcode{0001cpg.....0.....Z}
File: hiro.tex
Page:		 \setcounter{page}{586}
\begin{references}
\reference Bally, J., Stark, A. A., Wilson, R. W., \& Henkel,
C. 1988, \apj, 324, 223
\reference Ba\l uci\'{n}ska-Church, M., \& McCammon, D. 1992, \apj,
400, 699
\reference Binette, L., Dopita, M. A., D'Odorico, S., \&
Benvenuti, P. 1982, \aap, 115, 315
\reference Burke, B. E., Mountain, R. W., Harrison, D. C.,
Bautz, M. W., Doty, J. P., Ricker, G. R., \& Daniels, P. J. 1991, IEEE
Trans., ED-38, 1069
\reference Feldman, U. 1992, Phys. Scr, 46, 202
\reference
Gotthelf, E. 1996, The {{\em ASCA}} news (Greenbelt: NASA GSFC), 4, 31
\reference Koyama, K., Awaki, H., Kunieda, H., Takano,
S., Tawara, Y., Yamauchi, S., \& Nagase, F. 1989, nature, 339, 603
\reference Koyama, K., Maeda, Y., Sonobe, T., Takeshima, T.,
Tanaka, Y., \& Yamauchi, S. 1996, \pasj, 48, 249
\reference Lis, D. C., \& Goldsmith, P. F. 1989, \apj, 337, 704
\reference Lis, D. C., Goldsmith, P. F., Carlstrom, J. E., \&
Scoville N. Z. 1993, \apj, 402, 238
\reference Lis, D. C., \& Carlstrom, J. E. 1994, \apj, 424, 189
\reference Luck, R. E. 1982, \apj, 256, 177
\reference
Makishima, K., et al. 1996, \pasj, 48, 171
\reference
Ohashi, T., et al. 1996, \pasj, 48, 157
\reference Oka, T., Hasegawa, T., Hayashi, M., Hanada, T., \&
Sakamoto, S. 1998, \apj, 493, 730
\reference Ramirez, S. V., Sellgren, K., Terndrup, D. M.,
Carr, J. S., Balachadran, S., \& Blum, R. D. 1997, IAU
 Symp. 184, The Central Regions of the Galaxy and Galaxies,
 ed. Y.Sofue (London: Kluwer Academic Publishers), 67
\reference Ratag, M. A., Pottasch, S. R., Dennefeld,
 M., \& Menzies, J. W. 1992, \aap, 255, 255
\reference Reid, M. J., Schneps, M. H., Moran, J. M., Gwinn,
C. R., Genzel, R., Downes, D., \& R\"{o}nn\"{a}ng, B. 1988, \apj, 330,
809
\reference Sakano, M., Nishiuchi, M., Maeda, Y., Koyama,
K., \& Yokogawa, J. 1997, IAU
 Symp. 184, The Central Regions of the Galaxy and Galaxies,
 ed. Y.Sofue (London: Kluwer Academic Publishers), 443
\reference Sellgren, K., Carr, J. S., \& Balachandran,
S. C. 1997, IAU Symp. 184, The Central Regions of the Galaxy and
Galaxies, ed. Y.Sofue (London: Kluwer Academic Publishers), 21
\reference
Serlemitsos, P. J., et al. 1995, \pasj, 47, 105
\reference Shaver, P. A., McGee, R. X., Newton, L. M., Danks,
A. C., \& Pottasch S. R. 1983, \mnras, 204, 53
\reference Sunyaev, R., \& Churazov, E. 1998, \mnras, 297,1279
\reference
Tanaka, Y., Inoue, H., \& Holt S. S. 1994, \pasj, 46, L37
\reference Yamauchi, S., Kawada, M., Koyama, K., Kunieda,
H., Tawara, Y., \& Hatsukade, I. 1990, \apj, 365, 532

\end{references}
------------------------------------------------------------------------

\adsbibcode{0000asdi....0.....Z}
\setcounter{page}{297}
\title{Accretion History of Super-massive Black Holes}
\author{Priyamvada Natarajan}
\affil{Canadian Institute for Theoretical Astrophysics, 60 St. George
Street, Toronto M5S 3H8, Canada}

\begin{abstract}
We show that the luminosity function of the actively star-forming
Lyman break galaxies and the B-band quasar luminosity function at $z =
3$ can be fitted reasonably well with the mass function of collapsed
galaxy-scale dark matter halos predicted by viable variants of
hierarchical cold dark matter dominated cosmological models for
lifetimes $t_Q$ of the optically bright phase of QSOs in the range
$10^{6}$ to $10^{8}\,$yr. There is a strong correlation between $t_Q$
and the required degree of non-linearity in the relation between black
hole and host halo mass. Such a non-linear relation is motivated by
suggesting that the mass of supermassive black holes may be limited by
the back-reaction of the emitted energy on the accretion flow in a
self-gravitating disc. This would imply a relation of black hole to
halo mass of the form $M_{\rm bh} \propto v_{\rm halo}^5 \propto
M_{\rm halo}^{5/3}$ and a typical duration of the optically bright QSO
phase of the order of the Salpeter time, $\sim\,10^{7}\,$yr. The high
integrated local mass density of black holes inferred from recent
kinematic determinations of black hole masses in nearby galaxies seem
to indicate that the overall efficiency of supermassive black holes
for producing blue light is lower than was previously assumed. We
discuss three possible accretion modes with low optical emission
efficiency: (i) accretion well above the Eddington rate, (ii)
accretion obscured by dust, and (iii) accretion below the critical
rate leading to an advection dominated accretion flow lasting for a
Hubble time. We further argue that accretion with low optical
efficiency might be closely related to the origin of the hard X-ray
background.
\end{abstract}

\begin{references}

\reference Almaini, O., Boyle, B. J., Shanks, T., Griffiths, R. E., Roche, N.,
Stewart, G. C. \& Geogantopoulos, I. 1996, \mnras, {\bf 282}, 295

\reference Almaini, O., Lawrence, A. \& Boyle, B. J. 1998, \mnras, (submitted)

\reference Bagla, J. S. 1998, \mnras, {\bf 299}, 417

\reference Baugh, C. M., Frenk, C. S. \& Lacey, C. 1998, \apj, {\bf 498}, 504

\reference Begelman, M. C. 1978, \mnras, {\bf 184}, 53

\reference Bershady, M., Majewski, S. R., Koo, D. C., Kron, R. G. \& Munn, A.
1997, \apj, {\bf 490}, 41

\reference Blain, A. W., Smail, I., Ivison, R. \& Kneib, J-P. 1998, \mnras, (in
press) astro-ph/9806062

\reference Boyle, B. J., \etal\ 1988, \mnras, {\bf 297}, 53

\reference Cavaliere, A. \& Szalay, A. 1986, \apj, {\bf 311}, 589

\reference Chokshi, A. \& Turner, E. L. 1992, \mnras, {\bf 259}, 421

\reference Di Matteo, T. \& Fabian, A. C. 1997, \mnras, {\bf 286}, 393

\reference Efstathiou, G. P. \& Rees, M. J. 1988, \mnras, {\bf 230}, 5p

\reference Faber, S. M., \etal\ 1997, \aj, {\bf 114}, 1771

\reference Fabian, A. C., Barcons, X., Almaini, O. \& Iwasawa, K., 1998, \mnras,
{\bf 297}, 11

\reference Frenk, C. S., \etal\ 1997, in \PersicSalucci\ p.~335

\reference Fukugita, M., Hogan, C. J. \& Peebles, P. J. E. 1998, \apj, {\bf
503}, 518

\reference Genzel, R., Eckart, A., Ott, T. \& Eisenhauer, F. 1997, \mnras, {\bf
201}, 219

\reference Giavalisco, M., Steidel, C. S. \& Macchetto, F. D. 1996, \apj, {\bf
470}, 189

\reference Haehnelt, M., Natarajan, P. \& Rees, M. J. 1998, \mnras, (in press)

\reference Haehnelt, M. G. \& Rees, M. J. 1993, \mnras, {\bf 263}, 168 (HR93)

\reference Haiman, Z. \& Loeb, A. 1997, \apj, {\bf 503}, 505

\reference Hasinger, G. 1998, Astron. Nachr, {\bf 319}, 37

\reference Hasinger, G., Burg, R., Giacconi, R., Schmidt, J., Tr\"umper, J. \&
Zamaroni, G. 1998, \aap, {\bf 329}, 482

\reference Ho, L. C., Fillipenko, A. V. \& Sargent, W. L. W. 1997, \apjs, {\bf
112}, 315

\reference Jing, J. P. \& Suto, Y. 1998, \apjl, {\bf 494}, L5

\reference Kormendy, J. \& Richstone, D. 1995, \araa, {\bf 33}, 581

\reference Magorrian, J., \etal\ 1998, \aj, {\bf 115}, 2285 (Mag98)

\reference McHardy, \etal\ 1998, \mnras, {\bf 295}, 641

%\reference McMahon, Irwin \& Hazard 1997 \mn{No such paper}

\reference Miyoshi, M., Moran, M., Hernstein, J., Greenhill, L., Nakai, N.,
Diamond, P. \& Inoue, N. 1995, \nat, 373, 127

\reference Narayan, R. \& Yi, I. 1995, \apj, {\bf 452}, 710

\reference Peacock, J. A. 1997, astro-ph/9712068 % no more info on ADS

\reference Pettini, M., \etal\ 1997, \apj, {\bf 478}, 536

\reference Phinney, E. S. 1997, in \SandersBarnes

\reference Rees, M. J. 1984, \araa, {\bf 22}, 471

\reference Schmidt, M., Schneider, D. P. \& Gunn, J. E. 1994, \aj, {\bf 110}, 68

\reference Schmidt, M., \etal\ 1998, \aap, {\bf 329}, 495

\reference Shaver P., Wall, J. V., Kellermann, K. I., Jackson, C. A. \& Hawkins,
M. R. S. 1996, \nat, {\bf 384}, 439

\reference Small, T. A. \& Blandford, R. D. 1992, \mnras, {\bf 259}, 725

\reference Soltan, A. 1982, \mnras, {\bf 200}, 115.

\reference Steidel, C. S. \& Hamilton, D. 1992, \aj, {\bf 104}, 941

\reference Steidel, C. S., Pettini, M. \& Hamilton, D. 1995, \aj, {\bf 110},
2519

\reference Steidel, C. S., Giavalisco, M., Pettini, M., Dickinson, M. \&
Adelberger, K. L\ 1996, \apj, {\bf 462}, L17

\reference Steidel, C. S., Adelberger, K., Dickinson, M., Giavalisco, M. \&
Pettini, M. 1998, \apj, {\bf 492}, 428

\reference van der Marel, R. P. 1997, in \SandersBarnes\ astro-ph/9712076

\reference Wandel, A. 1991, \aap, {\bf 241}, 5

\reference Warren, S. J., Hewitt, P. C. \& Osmer, P. S. 1994, \apj, {\bf 421},
412

\reference Watson, W. D. \& Wallin, B. K. 1994, \apjl, {\bf 432}, L35

\reference Yi, I. \& Boughn, S. P. 1998, {\bf 499}, 198

-----------------------------------------------------------------------

\adsbibcode{0000egdg....0.....Z}
\begin{thebibliography}{}
\bibitem[a(1)]{zy1} Becklin E.E., Neugebauer G., 1968, ApJ 151, 145
\bibitem[a(1)]{zy2} Becklin E.E., Neugebauer G., 1975, ApJ 200, L71
\bibitem[a(1)]{zy3} Becklin E.E., Neugebauer G., 1978, PASP 90, 657
\bibitem[a(1)]{zy4} Diego F., 1985, PASP 97, 1209
\bibitem[a(1)]{zy5} Emerson D.T., Klein U., Haslam C.G.T., 1979, A\&A 76, 92
\bibitem[a(1)]{zy6} Kreysa E., Beeman J.W., Haller E.E., 1996 , ESA-SP 388
\bibitem[a(1)]{zy7} Mezger P.G., Duschl W.J., Zylka R., 1996, A\&AR, 7, 4
\end{thebibliography}




%R 0000scor....0.....Z
%Z \begin{thebibliography}{}

\bibitem[\protect\citeauthoryear{Cacciani \& Moretti}{1994}]

{CEM94}

Cacciani, A., Moretti, P.F., (1994), in

``Instrumentation in Astronomy VIII'',

{\bf 2198}, 219, ed. SPIE, Washington, USA

\bibitem[\protect\citeauthoryear{Cacciani \&  Moretti}{1995}]{CEM95}

Cacciani, A., Moretti, P.F. (1995), in

`` GONG'94: Helio- and asteroseismology from Earth and Space''

1995, {\bf 76}, 440,

R.K. Ulrich, E.J. Rhodes Jr. \& W. D\"appen  (eds.),

ASP Conference Series

\bibitem[\protect\citeauthoryear{Marmolino {\it et al.}}{1996}]{MOSS96}

Marmolino, C., Oliviero, M., Severino, G., and Smaldone L.A. (1996),

{\it A\&A}, submitted

\end{thebibliography}


%R 0001IAUS....0.....Z
%Z
\begin{thebibliography}{}
&#37;
&#37;\bibitem []{a} Bally, J., Morse, J., \& Reipurth B. 1996, in {\it Science with
&#37;       the Hubble Space Telescope---II}, eds.\ P. Benvenuti, F.~D. Macchetto,
&#37;       \& E.~J. Schreier, p.\ 491
\bibitem []{b} Burrows, C. et al.\ 1997, \apj, (in press)
&#37; \bibitem []{c} Hodapp, K.-W. \& Ladd, E.~F. 1995, \apj, 453, 718
\bibitem []{d} Bell, K.~R. \& Lin, D.~N.~C. 1994, \apj, 427, 987
\bibitem []{e} McCaughrean, M.~J., Rayner, J.~T., \& Zinnecker, H. 1994,
       \apj, 436, L189
\bibitem []{f} Miyoshi, M., Moran, J., Herrnstein, J., Greenhill, L.,
        Nakai, N., Diamond, P., Inoue, M.\ 1995, Nature, 373, 127
\bibitem []{g} Sargent, A.~I. \& McCaughrean, M.~J. 1997, in preparation
&#37;\bibitem []{h} Ray T.~P. 1996, in {\it Solar and Astrophysical MHD flows}, ed.
&#37;       K. Tsinganos (Kluwer), p999
\bibitem []{i}Reipurth, B., 1994, {\it A General Catalog of Herbig-Haro Objects},
       electronically published via anonymous ftp at {\tt ftp.hq.eso.org},
       directory \\ {\tt pub/Catalogs/Herbig-Haro}
\bibitem []{j} Suttner, G., Smith, M.~D., Yorke, H.~W., \& Zinnecker, H. 1997,
      \aap, (in press)
&#37;\bibitem []{k} Smith, M.~D., Suttner, G., \& Zinnecker, H.  1997, \aap, (in press)
&#37;\bibitem []{l} Smith, M.~D., Brand, P.~W.~J.~L. 1990, \mnras, 245, 108
\bibitem []{m} Wouterloot, J.~G.~A. \& Walmsley, C.~M. 1986, \aap, 168, 237
\bibitem []{n} Wouterloot, J.~G.~A., Henkel, C., \& Walmsley, C.~M. 1989,
      \aap, 215, 131
\bibitem []{o} Zinnecker, H., Bastien, P., Arcoragi, J.~P., \& Yorke, H.~W. 1992,
      \aap, 265, 726
\bibitem []{p} Zinnecker, H., McCaughrean, M.~J., \& Rayner, J.~T. 1996,
        in {\it Disks and Outflows around Young Stars}, eds.\ S. Beckwith,
        J. Staude, A. Quetz, and A. Nat\-ta, Lecture Notes in Physics 465,
        (Heidelberg: Springer), p.\ 236
&#37;
\end{thebibliography}


%R 0000ASPC....0.....Z
%J The Central Engine of Active Galactic Nuclei,
ASP Conference Series, Vol. 373, proceedings of the conference held
16-21 October, 2006 at Xi'an Jioatong University, Xi'an, China. Edited by
Luis C. Ho and Jian-Min Wang, p.651
%D 10/2007
%T Distribution and Kinematics of Molecular Gas in Perseus A
%A Ao, Y.-P.; Lim, J.; Trung, D.-V.
%B Here we report the latest results of the observation of the
molecular gas of Perseus A from the Submillimeter Array (SMA),
which provides the most direct evidence yet for the deposition of
molecular gas from an X-ray cooling flow.
%Z \reference {Conselice, C. J., Gallagher, J. S., \& Wyse, R. F. G. 2001, AJ, 122, 2281}\
\reference {Edge, A. C. 2001, MNRAS, 328, 762}\
\reference {Mathews, W. G., \& Brighenti, F. 2003, ARA\&A, 41, 191}\
%I ORIGINAL: YES

%R 0001ASPC....0.....Z
%J Multi-Spin Galaxies, ASP Conference Series, Vol. 486, Proceedings of a conference held 30 September-3 October 2013 at INAF-Osservatorio Astronomico
 di Capodimonte, Napoli, Italy. Edited by Enrichetta Iodice and Enrico Maria Corsini.  San Francisco: Astronomical Society of the Pacific, 2014., p.81
%D 05/2014
%T Structural Properties of Polar Ring Galaxy Candidates
%A Smirnova, K.; Moiseev, A.
%B We have considered polar ring galaxy candidates, the images of which can be found in the Sloan Digital Sky Survey. The sample of 78 galaxies includes the most reliable candidates from the Sloan Digital Sky Survey Polar Ring Galaxy Catalogue and Polar Ring Galaxy Catalogue, some of which already have kinematic confirmations. We analyze the distributions of the studied objects by the angle between the polar ring and the central disk, and by the optical diameter of the outer ring structures.
%Z \reference {Moiseev},
  A.~V. 2012, Astrophys. Bull., 67, 147
\reference {Moiseev}, A.~V., {Smirnova},
  K.~I., {Smirnova}, A.~A., \&amp; {Reshetnikov}, V.~P.  2011, MNRAS,
  418,~244
\reference {Whitmore}, B.~C., {Lucas}, R.~A., {McElroy}, D.~B., et al. 1990,
  \aj, 100, 1489
%I ORIGINAL: YES


%R 0001nla.....0.....Z
%J Proceedings of the NASA LAW 2006, held February 14-16, 2006, UNLV, Las Vegas.
Published by NASA Ames, Moffett Field, CA, 2006., p.140
%D 00/2006
%T Visible to Near Infrared Emission Spectra of Electron-Excited H_2
%A Aguilar, A.; James, G. K.; Ajello, J. M.; Abgrall, H.; Roueff, E.
%F AA(Jet Propulsion Laboratory, California Institute of
Technology, Pasadena, CA 91109 <EMAIL>alex.aguilar@jpl.nasa.gov</EMAIL>),
AB(Jet Propulsion Laboratory, California Institute of
Technology, Pasadena, CA 91109),
AC(Jet Propulsion Laboratory, California Institute of
Technology, Pasadena, CA 91109 <EMAIL>joseph.m.ajello@jpl.nasa.gov</EMAIL>),
AD(LUTH and UMR 8102 du CNRS, Observatorie de Paris, 92195
Meudon Cedex, France <EMAIL>herve.abgrall@obspm.fr</EMAIL>),
AE(LUTH and UMR 8102 du CNRS, Observatorie de Paris, 92195
Meudon Cedex, France)
%B The electron-impact induced fluorescence spectrum of H_2 at 100 eV
from 700~nm to 950~nm at a spectral resolution of between 0.2~nm to
0.3~nm has been measured. The laboratory spectrum has been compared
with our theoretical simulated spectrum obtained by calculating the
lines emission cross sections from the upper states of g symmetry
(EF, GK, HH, P, O ; I, R, J, S ) towards the states of u symmetry
(B, C, B', D) of H_2. The nine above Born-Openheimer g-upper
states have been coupled together as well as the four above
Born-Openheimer u-lower states. The comparison seems adequate with
few minor discrepancies.
%Z  \bibitem[(Abgrall {\em et al.} 1999)]{Abgrall1999} Abgrall H., Roueff E., Liu X., Shemansky D.E., James G.K., 1999, J. Phys. B, \textbf{32}, 3813.
\bibitem[(Abgrall {\em et al.} 2000)]{Abgrall2000}Abgrall H., Roueff E., Drira I. 2000, A\&A Suppl. Ser. \textbf{141}, 297-300
\bibitem[(Ajello {\em et al.} 2005)]{Ajello2005} Ajello J.M., Vatti-Palle, P., Abgrall H., Roueff E., Bhardwaj A., Gustin J., 2005, Ap. J. Supp., \textbf{159}, 314-330.
\bibitem[(Dalgarno {\em et al.} 1999)]{Dalgarno1999} Dalgarno A., Min Yan and Weihong Liu, 1999, Ap. J. Supp., \textbf{125}, 237.
\bibitem[(Dziczek {\em et al.} 2000)]{Dziczek2000} Dziczek D., Ajello J.M., James G.K., Hansen D.L., 2000, Phys. Rev. A, \textbf{61}, 64702.
\bibitem[(Giannini {\em et al.} 2004)]{Giannini2004} Giannini T., McCoey C., Caratti o Garatti A., Nisini B., Lorenzetti D., Flower D.R., 2004 ,A\&A, \textbf{419}, 999-1014.
\bibitem[(James {\em et al.} 1998)]{James1998} James G.K., Ajello J.M., Pryor W.R., 1998, J.Geophys. Res., \textbf{103}, 20113.
\bibitem[(Jonin {\em et al.} 2000)]{Jonin2000} Jonin C., Liu X., Ajello J.M., James G.K., Abgrall H., 2000, Ap. J. Supp., \textbf{129}, 247.
\bibitem[(Liu {\em et al.} 2002)]{Liu2002}Liu X., Shemansky D.E., Abgrall H., Roueff E., Dziczek D., Hansen D.L., Ajello J.M. 2002, Ap.J. Sup.Ser., \textbf{245},
229-245.
\bibitem[(Liu {\em et al.} 2003)]{Liu2003}Liu, X., Shemansky, D. E., Abgrall, H., Roueff, E., Ahmed, S. M., Ajello, J. M. 2003, J. Phys. B, \textbf{36},
173.
%\bibitem[(Nisini {\em et al.} 2005)]{Nisini2005} Nisini B., Bacciotti F., Giannini T., Massi F., Eisl\"{o}ffel J., Podio L., Ray T.P., 2005, A\&A, \textbf{441}, 159-170.
%\bibitem[(Liu {\em et al.} 1998)]{Liu1998} Liu et al. 1998, Ap. J., 103, 739.
%\bibitem[Liu et al. (2002)] Liu X., Shemansky D.E., Abgrall H., Roueff E., Dziczek D., Hansen D.L., Ajello J.M., 2002, Ap. J. Supp., \textbf{138}, 229.
%\bibitem[(Abgrall {\em et al.} 1997)]{Abgrall1997} Abgrall H., Roueff E., Liu X., Shemansky D.E., 1997, Ap. J., \textbf{481}, 557.
%I ORIGINAL: YES

%%%%%%%%%%%%%%%%%%%%%%%%%%%%%%%%%%%%%%%%%%%%%%%%%%%%%%%%%%%%%%%

\adsbibcode{0002dgeu....0.....Z}
%%PAGE 243


\title{Old-fashion Approach to Kinematically Decoupled Cores in
Elliptical Galaxies}
\author{Christian M. Boily}
\affil{Astronomisches Rechen-Institut, M\"{o}nchhofstrasse 12-14
Heidelberg D-69120, Germany\ e-mail: cmb@ari.uni-heidelberg.de }

\begin{abstract}

Collapsing cold spheroids are prone to surface and radial-orbit modes
of instability leading to triaxial configurations in equilibrium
(Merritt 1999). When the mass distribution is not reflected through
the equator, a net torque may act on the central region as the modes
grow. The twist sets the central part and the outer-reach of the
galaxy rotating in opposite direction.
 Can the equilibria show a dynamically decoupled central part,
 as observed for KDC ellipticals (eg, Carter et al. 1998; Hau et
al. 1999)?

To address this issue I have sought a stringent test for the
survival of streaming motion the above setup would give rise to.
 I built on an analytic model to find out when the accrued angular
momentum should be optimum: this is when a corrugated density mode
$\omega$ is of amplitude the initial aspect ratio or more
(eq. [1]). Analysis ends when the systems rebound, and numerical
simulations are used to probe the virialised equilibria.

\end{abstract}


\begin{references}

\reference Aarseth, S.J. \& Binney, J. 1978. \mnras, 185, 227
\reference Boily, C.M., Clarke,  C.J. \& Murray, S.D.  1999. \mnras,
302, 399 \label{BCM}
\reference Carter, D. et al. 1998. \mnras, 294, 182 \label{carter1998}
\reference Fellhauer, M. et al. 1999. New Astronomy (submitted 08/99) \label{fellhauer}
\reference Hau, G.K.T. et al. 1999. Astro-ph/9902170 (11 February 1999) \label{hau}
\reference Merritt, D. 1999. PASP, 111, 129 \label{merritt}

\end{references}


%%%%%%%%%%%%%%%%%%%%%%%%%%%%%%%%%%%%%%%%%%%%%%%%%%%%%%%%%%%%%%%

\adsbibcode{0003dgeu....0.....Z}
%%PAGE 279

\title{Observing a Merging Sequence of Luminous IR Galaxies}

\author{K. Y. Lo\altaffilmark{1}, C. Y. Hwang\altaffilmark{2},
S. W. Lee, D.-C. Kim, W. H. Wang, T. H. Lee}
\affil{Academia Sinica Institute of Astronomy \& Astrophysics,
    Taipei, Taiwan, ROC}

\author{R. Gruendl, Y. Gao\altaffilmark{3}}
\affil{Astronomy Department, University of Illinois, Urbana, IL 61801,
USA}

\begin{abstract}

To provide observations of what happens as two galaxies interact, we
have started an observational program to study how the ISM and star
formation vary along a merging sequence of interacting luminous IR
galaxies.  We use BIMA maps of the CO(1-0) emission, VLA HI and radio
continuum observations, ISOCAM 7 and 15 $\micron$ images, optical and
near-IR imaging to characterize the gas distribution and star
formation activities.  Preliminary results of this survey are
described, as well as detailed results on the active star forming
``interaction" region in the Antennae (Arp 244).  Finally, 15\% of a
complete sample of ULIG are found to consist of two galaxies separated
by 10 to 60 kpc, in apparent conflict with the conventional view that
ULIG are the final stages of the merging of two galaxies.

\end{abstract}

\keywords{molecular gas, interacting galaxies, luminous IR galaxies,
starbursts}

\begin{references}
\reference Adler, D., Allen, R. J., Lo, K. Y., 1991, \apj, 382, 475
\reference Bushouse, H., Telesco, C., Werner, M. 1998, \aj, 115, 938
\reference Genzel, R., Lutz, D., Tacconi, L. 1998, Nature, 395, 859
\reference Gao, Yu, Gruendl, R., Hwang, C. Y., Lo, K. Y. 1999, ``Galaxy
Interactions at Low and High Redshift," eds. J. Barnes \& D. Sanders, (Dordrecht: Kluwer), 227
\reference Hwang, C. Y., Lo, K.Y., Gao, Y., Gruendl, R., Lu, N-Y., 1999, \apjl, 511, L17
\reference Kim, D.-C. et al 1999, in preparation.
\reference Mihos, C., Hernquist, L. 1996 \apj, 464, 641
\reference Mirabel, F. et al, 1998, \aap, 333, L1
\reference Nikola, T. et al 1998, \apj, 504, 749
\reference Sanders, D., Mirabel, I. 1996, \araa, 34, 749
\reference Sanders, D., Scoville, N., Soifer, B. 1991, \apjl, 370, 158.
\reference Wang, W. H. et al, 1999, \apj, in revision.
\reference Whitmore, B., Schweizer, F. 1995, \aj, 109, 960

\end{references}


%R 0001fthp....0.....Z
%T Peculiar, Low Luminosity Type II Supernovae: Site of Black Hole Formation?
%A Zampieri, L.; Pastorello, A.; Turatto, M.; Cappellaro, E.; Benetti, S.; Altavilla, G.; Mazzali, P.; Hamuy, M.
%F AA(INAF -- Astronomical Observatory of Padova, Padova, Italy), AB(INAF -- Astronomical Observatory of Padova, Padova, Italy, and Department of Astronomy, University of Padova, Padova, Italy, and Department of Physics and Astronomy, University of Oklahoma, Norman OK, USA), AC(INAF -- Astronomical Observatory of Padova, Padova, Italy), AD(INAF -- Astronomical Observatory of Capodimonte, Napoli, Italy), AE(INAF -- Astronomical Observatory of Padova, Padova, Italy), AF(INAF -- Astronomical Observatory of Padova, Padova, Italy, and Department of Astronomy, University of Padova, Padova, Italy), AG(INAF -- Astronomical Observatory of Trieste, Trieste, Italy), AH(The Observatories of the Carnegie Institution of Washington, Pasadena CA, USA)
%J From Twilight to Highlight: The Physics of Supernovae.  Proceedings of the ESO/MPA/MPE Workshop held in Garching, Germany, 29-31 July 2002, p. 216.
%D 00/2003
%B A number of supernovae classified as Type II show remarkably peculiar
properties such as an extremely low expansion velocity and an
extraordinary small amount of <SUP>56</SUP> Ni in the ejecta. We have
modelled the available observations of these peculiar Type II
supernovae by means of a new semi-analytic light curve code. We find
that these events are under-energetic with respect to a typical Type
II supernova and that the inferred mass of the ejecta is large. These
supernovae are likely to originate from the explosion of a massive
progenitor in which the rate of early infall of stellar material on
the collapsed core is large. Events of this type could form a black
hole remnant, giving rise to significant fallback and late-time
accretion.

%Z \bibitem{arn82}
Arnett, W. D., 1980, ApJ, 237, 541

\bibitem{arfu89}
Arnett, W. D., Fu, A., ApJ, 1989, 340, 396

\bibitem{balbet al.00}
Balberg, S., Zampieri, L., Shapiro, S. L., 2000, ApJ, 541, 860

\bibitem{bet al.01}
Benetti, S. et al., 2001, MNRAS, 322, 361

\bibitem{cu00}
Chugai, N. N., Utrobin, V. P., 2000, A&amp;A, 354, 557

\bibitem{fr99}
Fryer, C. L., 1999, ApJ, 522, 413

\bibitem{hamet al.02}
Hamuy, M. et al., 2003, in preparation

\bibitem{pet al.02}
Pastorello, A. et al., 2002, MNRAS, submitted

\bibitem{pop92}
Popov, D. V., 1992, ApJ, 414, 712

\bibitem{tet al.98}
Turatto, M. et al., 1998, ApJ, 498, L129

\bibitem{zam-ww95}
Woosley, S. E., Weaver, T. A., 1995, ApJS, 101, 181

\bibitem{zampet al.98}
Zampieri, L., Colpi, M., Shapiro, S. L., Wasserman, I., 1998, ApJ, 505, 876

\bibitem{zsc98}
Zampieri, L., Shapiro, S. L., Colpi, M., 1998, ApJ, 502, L149

\bibitem{zet al.02a}
Zampieri, L. et al., 2002a, MNRAS, submitted

\bibitem{zet al.02b}
Zampieri, L. et al., 2002b, in preparation



%R 0002IAUS....0.....Z
%Z
\begin{thebibliography}{}

&#37;\parskip=0pt
\bibitem[]{a}
Choi, M., Evans, N.J., II, \& Jaffe, D.T.\ 1993, ApJ, 417, 624
&#37; massive outflow properties

&#37;\bibitem[]{b}
&#37;Fridlund, C.V.M., Liseau, R., \& Perryman, M.A.C. 1993, A\&A, 273, 601
&#37; HH29 in L1551 - spectrophotometric imaging in the visible.

\bibitem[]{c}
Hodapp, K-W. 1994, ApJS, 94, 615
&#37;K' imaging survey, observed G192.16

\bibitem[]{d}
Howard, E. 1996, Thesis work conducted at the University of
Rochester, Rochester, New York
&#37; massive outflow properties

&#37;\bibitem[]{e}
&#37;Hunter et al. 1996, Thesis work, in preparation
&#37; massive outflow properties

\bibitem[]{f}
Masson, C.R., \& Chernin, L.M.\ 1993, ApJ, 414, 230
&#37; general jet theory reference

\bibitem[]{g}
Moriarty-Schieven, G.H., Snell, R.L., Strom, S.E., \& Grasdalen,
G.L.\ 1987a, ApJ, 317, L95
&#37; CS obs of L1551 compared to CO low velocity emission.

\bibitem[]{h}
Moriarty-Schieven, G.H., Snell, R.L., Strom, S.E., Schloerb, F.P., \&
Strom, K.M.\ 1987b, ApJ, 319, 742
&#37; High res images of the co outflow in L1551

\bibitem[]{i}
Raga, A.C.\ 1995, Rev.\ Mex.\
A.\ A.\ Ser.\ Conf.\, 1, 103
&#37; review of jet theory from the Cozumel conference.

&#37;\bibitem[]{j}
&#37;Raga, A.C., Mundt, R. \& Ray, T.P. 1991, A\&A, 252, 733
&#37; collimation of stellar jets - [SII] imaging of the IRS5 jet in L1551

\bibitem[]{k}
Reipurth, B.\ 1994, A general catalog of Herbig--Haro objects

\bibitem[]{l}
Shepherd, D.S., \& Churchwell, E.B. 1996, ApJ 472, 225

\bibitem[]{m}
Shepherd, D.S., Sargent, A.I., \& Churchwell, E.B. 1997, in preparation

\bibitem[]{n}
Stocke, J.T., Hartigan, P.M., Strom, S.E., Strom, K.M., Anderson,
E.R., Hartmann, L.W., \& Kenyon, S.J.\ 1988, ApJS, 68, 229
&#37; overall review of the L1551 outflow and jet analysis

\end{thebibliography}



%R 0003IAUS....0.....Z
%Z
\begin{thebibliography}{}

\bibitem[Andr\'e (1995)]{a95}
Andr\'e P. 1995, Ap\&SS 224,29
\bibitem[Andr\'e (1996)]{a96}
Andr\'e, P. 1996, ASP Conf. Ser., 93,273
\bibitem[Andr\'e et al. (1990)]{amdm}
Andr\'e et al. 1990, A\&A 236, 180
\bibitem[Andr\'e, Ward-Thompson \& Barsony (1993)]{awb}
Andr\'e, Ward-Thompson, Barsony 1993, ApJ406,122
\bibitem[Anglada (1996)]{angla96}
Anglada G. 1996, ASP Conf. Ser., 93,3
\bibitem[Bachiller (1996)]{bac96}
Bachiller R. 1996, Ann.Rev.A\&A 34,111
\bibitem[Bontemps, Andr\'e \& Ward-Thompson (1995)]{baw}
Bontemps, Andr\'e, Ward-Thompson 1995, A\&A 297,98
\bibitem[Bontemps et al. (1996a)]{batc}
Bontemps et al. 1996a, A\&A 311,858
\bibitem[Bontemps et al. (1996b)]{bwa}
Bontemps,Ward-Thompson,Andr\'e 1996b A\&A314,477
\bibitem[]{a}
B\"uhrke, Mundt, Ray 1988, A\&A 200,99
\bibitem[]{b}
Curiel S. 1995, RevMexAA (Serie de Conf.) 1,59
\bibitem[]{c}
Curiel et al. 1993, ApJ 415,191
\bibitem[Dent et al. (1995)]{dent95}
Dent et al. 1995, MNRAS 277,183
\bibitem[Leous et al. (1991)]{lfam}
Leous et al. 1991, ApJ 379,683
\bibitem{mckeeholl87}
McKee \& Hollenbach 1987, ApJ 322,275
\bibitem{raga95}
Raga A.C. 1995, RevMexAA (Serie de Conf.) 1,103
\bibitem[]{d}
Raga, Binette, Cant\'o 1990, ApJ 360,612
&#37;\bibitem[Raga \& Kofman (1992)]{rag-kof92}
&#37;Raga \& Kofman 1992, ApJ 386,222
\bibitem{rodrig95}
Rodr\'\i guez L.F. 1995, RevMexAA (Serie de Conf.) 1,1

\end{thebibliography}


%R 0004IAUS....0.....Z
%Z
\begin{thebibliography}{}

\bibitem[Bally \& Lada 1983]{b1}
Bally, J., \& Lada, C. 1983, \apj, 265, 824

\bibitem[b2]{b2}
Beichman, C. A., Becklin, E. E., \& Wynn-Williams, C. 1979, \apjl, 232, L47

\bibitem[b16]{b16}
Bessel, M. S., \& Brett, J. M. 1988, \pasp, 100, 1134

\bibitem[b3]{b3}
Black, J.~H., \& Dalgarno, A. 1976, \apj, 203, 132


\bibitem[b4]{b4}
Black, J.~H.,  \& van Dishoeck, E. F. 1987, \apj, 322, 412

\bibitem[b5]{b5}
Heyer, M.~H., Snell, R. L., Morgan, J., \& Schloerb, F. P. 1989, \apj, 346, 220

\bibitem[b6]{b6}
Hollenbach, D., \& McKee, C. F. 1989, \apj, 342, 306

&#37;\bibitem[b7]{b7}
&#37;Hillenbrand, L. A., Meyer, M. R., Strom, S. E., \& Skrutskie,
&#37;M. F. 1995, \aj, 109, 280

\bibitem[b8]{b8}
Jaffe, D.~T., Davidson, J.~A., Dragovan, M., \& Hildebrand, R.~H. 1984, \apj, 284, 637

&#37;\bibitem[b9]{b9}
&#37;Koornneef, J. 1983, \aap, 128, 84

\bibitem[b10]{b10}
Kurtz, S., Churchwell, E., \& Wood, D. O. S. W. 1994 \apjs, 91, 659

\bibitem[b11]{b11}
Kwan, J. 1977, \apj, 216, 713

\bibitem[b15]{b15}
Meyer, M. R. 1996, Ph.D. Thesis, University of Massachusetts

&#37;\bibitem[]{}
&#37;Richardson, K.~J., White, G.~J., Gee, G., Griffin, M.~J., Cunningham, C.~T., Ade, P.~A.~R., \& Avery, L. W. 1985, \mnras, 216, 713

\bibitem[b12]{b12}
Ruiz, A., Rodr\'{\i}guez, L. F., Cant\'o, J., \& Mirabel, I.~F. 1992, \apj, 398, 139

\bibitem[b13]{b13}
Snell, R.~L., \& Bally, J. 1986, \apj, 303, 683

\bibitem[b14]{b14}
Tamura, M., Gatley, I., Joyce, R.~R., Ueno, M., Suto, H.,
 \& Sekiguchi, M. 1991, \apj, 378, 611


\end{thebibliography}


%R 0005IAUS....0.....Z
%Z
\begin{thebibliography}{}

\bibitem{b1}
Bachiller R., Cernicharo J., Martin-Pintado J., Tafalla M.,
Lazareff B., 1990, {\aap} 231, 174
\bibitem{b2}
Bachiller R.,  Mart\'{\i}n-Pintado J., Fuente A., 1991 {\aap} 243, L21
\bibitem{b3}
Bachiller R., Mart{\i}n-Pintado J., Fuente A., 1993, {\apj} 417, L45
\bibitem{b4}
Bachiller R., Guilloteau S., Dutrey A., Planesas P.,
   Mart{\i}n-Pintado J., 1995, {\aap} 299, 857
\bibitem{b5}
Bally J., Lada C.J., Lane, A.P., 1993 {\apj}  418, 322
\bibitem{b6}
Blake, G.A., Sutton E.C., Masson C.R., Phillips T.G., 1995, {\apj} 441, 689
\bibitem{c1}
Chernin, L.M., Masson, C.R., Gouveia Dal Pino, E.M., Benz, W., 1994, {\apj}
     426, 204
\bibitem{d1}
De Campli W.M., 1981, {\apj} 244, 124
\bibitem{g1}
Guilloteau, S., Bachiller, R., Fuente, A., Lucas, R., 1992, {\aap} 265, L49
\bibitem{m1}
Mart\'{\i}n-Pintado J., Bachiller R., Fuente A., 1992, {\aap} 254, 315
\bibitem{m2}
Mikami H., Umemoto T., Yamamoto S., Saito S., 1992, {\apj}  392, L87
\bibitem{r1}
Raga A.C., Taylor S.D., Cabrit S., Biro S., 1995, {\aap} 296, 833
\bibitem{s1}
Stahler S.W., 1993, in Astrophysical Jets, eds.: M. Livio, C. O'Dea
       \&\ D. Burgarella. Cambridge University Press, Cambridge
\bibitem{s2}
Suttner G., Smith M.D., Yorke H.W., Zinnecker H., 1996, {\aap} in press
\end{thebibliography}


\adsbibcode{0000AN......0.....Z}
}
\bibitem
\bibitem{aands}    Abramowitz, M., Stegun, I.A.: 1964, {\it Handbook of
 Mathematical Functions}, Dover, New York
\bibitem{aschwand} Aschwanden, M.J., Newmark, J.S., Delaboudini\`ere, J.-P.,
 Neupert,
 \newpage
 W.M.,
 Klimchuk, J.A., Gary, G.A.,  Fabrice, P.-F., Zucker, A.: 1999, ApJ 515, 842
 %\newpage
\bibitem{mist}     Ballai, I.: 2007, SoPh, in press
\bibitem{ist1}     Ballai, I., Erd\'elyi, R., Pint\'er, B.: 2005, ApJ 633, L145
\bibitem{baner}    Banerjee, D., Erd\'elyi, R., Oliver, R., O'Shea, E: 2007, SoPh, submitted
\bibitem{depontieu}De Pontieu, B., Erd\'elyi, R.: 2006, RSPTA 364, 383
\bibitem{edrob}    Edwin, P.M., Roberts, B.: 1983, SoPh 88, 179
\bibitem{erdelyi1} Erd\'elyi, R.: 2006, RSPTA 364, 289
\bibitem{erdelyi2} Erd\'elyi, R., Ballester, J.L., Ruderman, M.S.: 2007, SoPh, submitted
\bibitem{kosozhar} Kosovichev, A.G., Zharkova, V.V.: 1998, Nature 393, 317
\bibitem{moreton}  Moreton, G.E., Ramsey, H.E.: 1960, PASP 72, 357
\bibitem{nakariako}Nakariakov, V.M., Ofman, L., Deluca, E.E., Roberts, B., Davila, J.M.: 1999, Sci 285, 862
\bibitem{parker}   Parker, E.N.: 1978, ApJ 221, 368
\bibitem{thompson} Thompson, B.J., Gurman, J.B., Neupert, W.M., et al.: 1999, ApJ 517, L151
\bibitem{Wentzel}  Wentzel, D.G.: 1979, ApJ 227, 319


\adsbibcode{0000BaltA...0.....Z}
\refb Barentsen G., Farnhill H. J., Drew J. E. et al. 2014, MNRAS, 444,
3230

\refb Cardelli J. A., Clayton G. C., Mathis J. S. 1989, ApJ, 345, 245

\refb Dambis A. K. 2009, MNRAS, 396, 553

\refb Dambis A. K., Glushkova E. V., Berdnikov L. N., Joshi Y. C., Pandey A. K.
        2017, MNRAS, 465, 1505

\refb Dias W. S., Alessi B. S., Moitinho A., L\'{e}pine J. R. D. 2002,
A\&A, 389, 871; {\it Open Clusters and Galactic Structure} - version
3.4, available from http://www.astro.iag.usp.br/ocdb

\refb Drew J. E., Greimel R., Irwin I. M. et al. 2005, MNRAS, 362, 753

\refb Girardi L., Bertelli G., Bressan A. et al. 2002, A\&A, 391, 195

\refb Glushkova E. V., Koposov S. E., Zolotukhin I. Yu. et al. 2010,
Astron.  Lett., 36, 75

\refb Kharchenko N. V., Piskunov A. E., Schilbach E., Roser S., Scholz
R.-D. 2013, A\&A, 558, 53

\refb Koposov S. E., Glushkova E. V., Zolotukhin I. Yu. 2008, A\&A, 486,
771

\refb Lawrence A., Warren S. J., Almaini O. et al. 2007, MNRAS, 379, 1599

\refb McBride V. A, Coe M. J., Negueruela I., Schurch M. P. E., McGowan
K. E. 2008, MNRAS 388, 1198

\refb Monet D. G., Levine S. E., Canzian B. et al. 2003, AJ, 125, 984

\refb Sanders W. L. 1971, A\&A, 14, 226

\refb Skrutskie M. F., Cutri R. M., Stiening R. et al. 2006, AJ, 131,
1163

\refb Wright E. L., Eisenhardt P. R. M., Mainzer A. K. et al. 2010, AJ,
140, 1868

\refb Zacharias N., Finch C. T., Girard T. M. et al. 2013, AJ, 145, 144

\refb Zacharias N., Finch C., Subasavage J. et al. 2015, AJ, 150, 101


%R 0000BASI....0.....Z
%J Bulletin of the Astronomical Society of India, Vol. 42, pp. 165-183
%D 09/2014
%T A new three-band, two beam astronomical photo-polarimeter
%A Srinivasulu, G.; Raveendran, A. V.; Muneer, S.; Mekkaden, M. V.; Jayavel, N.; Somashekar, M. R.; Sagayanathan, K.; Ramamoorthy, S.; Rosario, M. J.; Jayakumar, K.
%F AA(Indian Institute of Astrophysics, Bangalore~560034, India)
AB(399, "Shravanam", 2^{nd} Block, 9^{th} Phase, J P Nagar, Bangalore~560108, India)
AC(Indian Institute of Astrophysics, Bangalore~560034, India)
AD(No 82, 17E Main, 6^{th} Block, Koramangala, Bangalore~560095, India)
AE(No 22, Bandappa lane, New Byappanahalli, Bangalore~560038, India)
AF(Indian Institute of Astrophysics, Bangalore~560034, India)
AG(Indian Institute of Astrophysics, Bangalore~560034, India)
AH(Indian Institute of Astrophysics, Bangalore~560034, India)
AI(210, 4^{th} Main, Lakshmi Nagar Extn, Porur, Chennai~600116, India)
AJ(24, Postal Nagar, Ampuram, Vellore-632009, India)
%B We designed and built a new astronomical photo-polarimeter that
can measure linear polarization simultaneously in three spectral bands.
It has a Calcite beam-displacement prism as the analyzer. The ordinary
and extra-ordinary emerging beams in each spectral bands are
quasi-simultaneously detected by the same photomultiplier by using a high speed
rotating chopper.
 A rotating superachromatic Pancharatnam halfwave
plate is used to modulate the light incident on the analyzer. The spectral
bands are isolated using appropriate dichroic and glass filters.

 We show that the reduction of 50% in the efficiency of
 the polarimeter because of the fact that the intensities of the two beams
are measured alternately is partly compensated by the reduced time to be
spent on the observation of the sky background. The use of a beam-displacement
 prism as the analyzer completely removes the polarization of
background skylight, which
is a major source of error during moonlit nights, especially, in
the case of faint stars.

The field trials that were carried out by observing several
polarized and unpolarized stars show the performance of the polarimeter
to be satisfactory.

%K instrumentation: polarimeters, techniques: polarimetric, methods: observational, data analysis

%Z \bibitem[Ashok et al. (1999)Ashok et al.]{asho} {Ashok N. M., Chandrasekhar T., Bhatt H. C., Manoj P., 1999, IAUC 7103}

\bibitem [Bessel (1979)Bessel]{bess79}{Bessel M. S., 1979, PASP, 91, 589}

\bibitem [Bessel (1993)Bessel]{bess93} {Bessel M. S., 1993, in  Stellar Photometry $-$
         Current Techniques and Future Developments, IAU Coll. No. 136, Ed. C J. Butler, I. Elliott, Cambridge University Press }

\bibitem [Deshpande et al. (1985)Deshpande et al.]{desh}
         {Deshpande M. R., Joshi U. C., Kulshrestha A. K., Bansidhar V. N. M.,
          Majumdar H. S., Pradhan S. N., Shah C. R., 1985,  BASI,  13, 157}

\bibitem [Frecker \& Serkowski (1976)Frecker \& Serkowski]{frec}
         {Frecker J. E., Serkowski K., 1976,  Appl. Optics, 15, 605}

\bibitem [Hough et al. (1991)Hough, Peacock \& Bailey]{houg}
          {Hough, J. H., Peacock, T. \& Bailey, J. A., 1991, MNRAS 248, 74}

\bibitem[Hsu \& Breger (1982)Hsu \& Breger]{hsu}
          {Hsu J., Breger M., 1982, ApJ, 262, 732}

\bibitem [Jain \& Srinivasulu (1991)Jain \& Srinivasulu]{jain}
          {Jain S. K., Srinivasulu G., 1991,  Opt. Eng., 30, 1415}

\bibitem[Kameswara Rao \& Raveendran (1993)Kameswara Rao \& Raveendran]{kame}
           {Kameswara Rao N., Raveendran A. V., 1993, A\&A, 274, 330}

\bibitem [Kikuchi (1988)Kikuchi]{kiku}
          {Kikuchi S., 1988,  Bull. Tokyo Astron. Obs., Second Series No. 281, 3267}

\bibitem [Kopal (1959)Kopal]{kopa} {Kopal Z., 1959, Close Binaries, Chapman and Hall, London}

\bibitem [Magalhaes et al. (1984)Magalhaes, Benedetti \& Roland] {maga84}
          {Magalhaes A. M., Benedetti E., Roland E. H., 1984, PASP, 96, 383}

\bibitem [Magalhaes \& Velloso (1988)Magalhaes \& Velloso]{maga88}
          {Magalhaes A. M., Velloso, W. F., 1988, in  Polarized
          Radiation of Circumstellar Origin, Eds: G.V. Coyne et al.,
           Vatican Observatory, Vatican City State, p. 727}

\bibitem[Manoj et al. (2002)Manoj, Maheswar \& Bhatt]{mano}
          {Manoj P., Maheswar G., Bhatt H. C., 2002, BASI, 30, 657}

\bibitem[Mathewson \& Ford (1970)Mathewson \& Ford] {math}
         {Mathewson D.~S., Ford, V.~L., 1970, Mem. Royal astr. Soc.,  74, 139}

\bibitem[Mekkaden (1999)Mekkaden]{mekk}{Mekkaden M. V., 1999, A\&A, 344, 111}

\bibitem[Parthasarathy et al. (2000)Parthasarathy, Jain \& Bhatt]{part}
        {Parthasarathy M., Jain S. K., Bhatt H. C., 2000, A\&A, 355, 221}

\bibitem [Piirola (1973)Piirola]{piir} {Piirola V., 1973, A\&A, 27, 382}

\bibitem[Raveendran (1999)Raveendran]{rave99} {Raveendran A. V., 1999, MNRAS, 303, 595}

\bibitem[Raveendran (2002)Raveendran]{rave02} {Raveendran A. V., 2002, MNRAS, 336, 992}

\bibitem[Raveendran et al. (2015)Raveendran et al.]{rave15}
          {Raveendran A. V., Srinivasulu G., Muneer S., et al.,
           2015, A stellar photo-polarimeter, Technical Report, Indian Institute of Astrophysics, Bangalore 560034}

\bibitem [Scaltriti et al. (1989)Scaltriti et al.]{scal}
          {Scaltriti F., Cellino A., Anderlucci E., Corcione L., Piirola V., 1989, MmSAI, 60, 243}

\bibitem [Schwarz \& Piirola (1999)Schwarz \& Piirola]{schw}
          {Schwarz H. E., Piirola V., 1999, {\it Nordic Optical Telescope Operating Manual No. 2}}

\bibitem [Serkowski (1974a)Serkowski]{serk1}
         {Serkowski K., 1974a, in  Methods of Experimental Physics,  12 Part A, Eds M.L.Meeks, N.P.Carleton, Academic
           Press, New York, p. 361}

\bibitem [Serkowski (1974b)Serkowski]{serk2}
         {Serkowski, K., 1974b, in Planets, Stars and Nebulae
Studied with Photopolarimetry, Ed. T.Gehrels, Tucson, University of
Arizona Press, p. 135}

\bibitem [Serkowski et al. (1975)Serkowski, Mathewson \& Ford]{serk}
         {Serkowski K., Mathewson D. S., Ford V. L., 1975, ApJ, 196, 261}

\bibitem [Young (1967)Young]{youn}{Young A. T., 1967, AJ, 72, 747}



\adsbibcode{0000PASA....0.....Z}

\reference Amram, P., Boulesteix, J., Georgelin, Y. M., et al. 1991, The Messenger, 64, 44

\reference Brand, J., \& Blitz, L. 1993, A\&A, 275, 67

\reference Caswell, J. L., \& Barnes, P. J. 1983, ApJ, 271, L55

\reference Caswell, J. L., \& Haynes, R. F. 1987, A\&A, 171, 261

\reference Chu, Y. H. 1982, ApJ, 254, 578

\reference Cohen, M., \& Barlow, M. J. 1980, ApJ, 238, 585

\reference Consid\`ere, S., \& Athanassoula, E. 1982, A\&A, 111, 28

\reference Consid\`ere, S., \& Athanassoula, E. 1988, A\&AS, 76, 365

\reference Crampton, D., \& Georgelin, Y. M. 1975, A\&A, 40, 317

\reference Deharveng, L., \& Maucherat, M. 1974, A\&A, 34, 465

\reference Forbes, D. 1988, A\&AS, 77, 439

\reference Georgelin, Y. M., Amram, P., Georgelin, Y. P., le Coarer, E., \& Marcelin, M. 1994, A\&AS, 108, 513

\reference Georgelin, Y. P., \& Georgelin, Y. M. 1971, A\&A, 12, 482

\reference Georgelin, Y. M., \& Georgelin, Y. P. 1976, A\&A, 49, 57

\reference Grabelsky, D. A., Cohen, R. S., Bronfman, L., \& Thaddeus, P. 1988, ApJ, 331, 181

\reference Gregory, P. C., Scott, W. K., \& Douglas, K. 1996, ApJS, 103, 427

\reference Griffith, M. R., \& Wright, A. E. 1993, AJ, 105, 1666

\reference Griffith, M. R., Wright, A. E., Burke, E. F., \& Ekers, R. D. 1995,
ApJS, 97, 347

\reference Haynes, R. F., Caswell, J. L., \& Simons, L. W. J. 1978,
Aust. J. Phys. Astrophys. Suppl., 45, 1

\reference Herbig, G. H. 1974, PASP, 86, 604

\reference Hodge, P. W., \& Kennicut, R. C. 1983, AJ, 88, 296

\reference Jourdain de Muizon, M. 1987, PhD thesis, Universit\'e de
Paris VII

\reference Kogure, T., et al. 1982, Contribution from the Departement of Astronomy, University of Kyoto, No. 133, ISSN 0388-0230

\reference le Coarer, E., Amram, P., Boulesteix, J., et al. 1992, A\&A, 257, 389

\reference Longmore, A. J., Clark, D. H., \& Murdin, P. 1977, MNRAS, 181, 541

\reference Mader, S., Zealey, W. S., Parker Q. A., \& Masheder M. 1999, MNRAS, 310, 331

\reference Marcelin, et al. 1994, Vol. 71 IAU Colloquium 149, Marseille

\reference Marcelin, M., Amram, P., Bartlett, J. G., Valls-gabaud, D., \& Blanchard, A. 1998, A\&A, 338, 1

\reference Oey, M. S., \& Massey, P. 1994, ApJ, 425, 635

\reference Parker, Q. A., \& Bland-Hawthorn, J. 1998, PASA, 15, 33

\reference Parker, Q. A., \& Phillipps, S. 1998a, PASA, 15, 28

\reference Parker, Q. A., \& Phillipps, S. 1998b, A\&G, 39, 10

\reference Parker, Q. A. , Phillipps, S., et al. 1999, MNRAS, in preparation

\reference Parker, Q. A., Russeil, D., \& Hartley, M. 1999, in preparation

\reference Pottash, S. R. 1965, Vistas in Astronomy, 6, 149

\reference Rizzo, J. R., \& Arnal, M. 1998, A\&A, 332, 1025

\reference Rodgers, A. W., Campbell, C. T., \& Whiteoak, J. B. 1960, MNRAS,
121, 103

\reference Rosado, M., Laval, A., le Coarer, E., Georgelin, Y. P., et al. 1996, A\&A, 308, 588

\reference Russeil, D., Georgelin, Y. M., Georgelin, Y. P., le Coarer, E., \& Marcelin, M. 1995, A\&AS, 114, 557

\reference Russeil, D. 1997, A\&A, 319, 788

\reference Russeil, D. 1998, PhD thesis, Universit\'e de Provence

\reference Tenorio-Tagle, G. 1979, A\&A, 71, 59

\reference Tenorio-Tagle, G., \& Bodenheimer, P. 1988, ARA\&A, 26, 145

\reference Walker, A. R., Zealey, W. S, \& Parker, Q. A. 1999, PASA, submitted

\reference Whitelock, P. A. 1985, MNRAS, 213, 59

\reference Whiteoak, J. B. Z., \& Green, A. J. 1996, A\&A, 118, 329

\reference Wright, A. E., Gregory, P. C., Burke, B. F., \& Ekers, R. D. 1994, ApJS, 91, 111
